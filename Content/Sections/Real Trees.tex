The continuum analogue of discrete trees are given by $\R$-trees. A survey paper on $\R$-trees can be found in \myworries{insert Le Gall reference}. An $\R$-tree is a compact metric space $(\cT, d)$ such that for every $a, b \in \cT$ the following two properties hold:
\begin{enumerate}
    \item There exists a unique isometry $i_{a, b} : [0, d(a, b)] \to \cT$ such that $i_{a, b}(0) = a$ and $i_{a, b}(d(a, b)) = b$.
    \item If $q: [0, 1] \to \cT$ is any continuous map such that $q(0) = a$ and $q(1) = b$ then the image of $q$ is the same as the image of $i_{a, b}$.
\end{enumerate}
Let $\pathbtw{a, b}$ denote the image of $i_{a, b}$. This is the unique path between $a$ and $b$.

$\R$-trees are often encoded by continuous excursions which can be seen as a continuous analogue of the height function of a tree. Let $f: [0, \sigma] \to [0, \infty)$ be a continuous excursion, meaning $f$ is continuous, $f(0) = f(\sigma) = 0$ and $f(x) > 0$ for all $x \in (0, \sigma)$. Using $f$ we can define a pseudo-metric
\begin{equation*}
    d_f(x, y) = f(x) + f(y) - 2 \min_{s \in [x \wedge y, x \vee y]} f(s).
\end{equation*}
Using this we can define the quotient space
\begin{equation*}
    \cT_f = [0, \sigma] / \{d_f = 0\}.
\end{equation*}
The space $\cT_f$ equipped with the metric $d_f$ is the $\R$-tree encoded by the excursion $f$. Let $p_f: [0, \sigma] \to \cT_f$ be the natural projection function. Then $\cT_f$ inherits a distinguished root vertex $\rho = p(0) = p(\sigma)$.