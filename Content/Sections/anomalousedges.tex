\subsection{The convergence of the out-forest holds conditionally on the multigraph being simple}
We will now show that the parts of the directed multigraph we observe far beyond the timescale of interest are with high probability simple. We will then use an argument by Joseph \cite{josephComponentSizesCritical2014} to show that this implies that \cref{prop:convoutforest} holds conditional on the resulting multigraph being simple. We let $B_n(k)$ be the number of `bad edges' up to time $k$; to be precise, it equals be the number of self-loops and edges created parallel to an existing edge in the same direction as that edge, up until discovery of the $k$th vertex of the out-forest. Following \cite{conchon--kerjanStableGraphMetric2020}, we call these anomalous edges. 
\begin{proposition}\label{prop.anomalousedges}
Suppose $\beta<1$. Then we have
$$\P\left(B_n(\lfloor n^\beta \rfloor)>0\right)\to 0$$
as $n\to \infty$.
\end{proposition}
\begin{remark}
We adapt the proof of \cite[Lemma 7.1]{josephComponentSizesCritical2014} and of \cite[Proposition 5.3]{conchon--kerjanStableGraphMetric2020} to the directed setting. A significant complication is caused by the conditioning on $$\left\{\sum_{i=1}^n D^-_i=\sum_{i=1}^n D^+_i\right\}.$$ We observe that in both papers, the proof of the aforementioned result is not fully correct, because the authors use the wrong expression for the probability of sampling an anomalous edge. However, the argument below can be adapted to the setting of \cite{josephComponentSizesCritical2014} and \cite{conchon--kerjanStableGraphMetric2020} to yield a correct proof.
\end{remark}
\begin{proof}
We distinguish between the following types of anomalous edges.\\
Self-loops occur when the out-half-edge of a vertex is paired to an in-half-edge of the same vertex.  Let $B^1_n(k)$ be the number of self-loops that are found up to time $k$. For $v$ explored up to time $\lfloor n^\beta\rfloor$, a vertex with in-degree $d^-_v$ and out-degree $d^+_v$, there are $d^-_v d^+_v$ possible combinations of an in-half-edge and an out-half-edge that form a self-loop connected to $v$. Any of these combinations of half-edges is paired with probability bounded above by 
$$\frac{1}{\sum_{i=\lfloor n^\beta \rfloor+1}^n\hat{D}^-_i}.$$
Parallel edges occur when an out-half-edge of a vertex is paired to an in-half-edge of one of its previously explored children. Let $B^2_n(k)$ be the number of parallel edges that are found up to time $k$. For any vertex $v$ with in-degree $d^-_v$, and a parent $p(v)$ with out-degree $d^+_{p(v)}$, there are at most $d^-_v d^+_{p(v)}$ possible combinations of an in-half-edge and an out-half-edge that form a parallel edge from $p(v)$ to $v$. Again, any of these combinations of half-edges is paired with probability bounded above by 
$$\frac{1}{\sum_{i=\lfloor n^\beta \rfloor+1}^n \hat{D}^-_i}.$$
The last type of anomalous edges is a surplus edge with multiplicity greater than 1. Let $B^3_n(k)$ be the number of surplus edges with multiplicity greater than 1 that are found up to time $k$. For a vertex $w$ with out-degree $d^+_w$ and a vertex $v$ with in-degree $d^-_v$, a multiple surplus edge from $w$ to $v$ can only occur if $v$ is discovered before $w$. In that case, there are at most $(d^+_w)^2(d^-_v)^2$ possible pairs of combinations of half-edges, and each of these pairs appears with probability bounded above by
$$\left(\frac{1}{\sum_{i=\lfloor n^\beta \rfloor+1}^n \hat{D}^-_i}\right)^2.$$
Let $p(i)$ denote the index of the parent of the vertex with index $i$. Also, denote $$\cG^n=\sigma\left(\hat{D}^-_1,\hat{D}^+_1,\dots,\hat{D}^-_n,\hat{D}^+_n \right).$$ Then, by the conditional version of Markov's inequality, 

\begin{align*}\P\left(\left.B^1_n(\lfloor n^\beta \rfloor)>0\right| \cG^n \right)&\leq \frac{\sum_{i=1}^{\lfloor n^\beta \rfloor} \hat{D}^-_i\hat{D}^+_i}{\sum_{i=\lfloor n^\beta \rfloor+1}^n \hat{D}^-_i}\wedge 1,\\
\P\left(\left.B^2_n(\lfloor n^\beta \rfloor)>0\right| \cG^n \right)&\leq \frac{\sum_{i=1}^{\lfloor n^\beta \rfloor} \hat{D}^-_i\E\left[\left.\hat{D}^+_{p(i)}\right|\cG^n\right]}{\sum_{i=\lfloor n^\beta \rfloor+1}^n \hat{D}^-_i}\wedge 1,\\
\P\left(\left.B^3_n(\lfloor n^\beta \rfloor)>0\right| \cG^n \right)&\leq \frac{\sum_{i=1}^{\lfloor n^\beta \rfloor}\sum_{j<i} (\hat{D}^+_i)^2 (\hat{D}^-_j)^2 }{\left(\sum_{i=\lfloor n^\beta \rfloor+1}^n \hat{D}^-_i\right)^2 }\wedge 1,\end{align*}
where we note that $p(i)$ is not adapted to $\cG^n$, because ancestral relations in the tree also depend on the surplus edges. However, we observe that by the Cauchy-Schwarz inequality,
\begin{align*}\sum_{i=1}^{\lfloor n^\beta \rfloor} \hat{D}^-_i\E\left[\left.\hat{D}^+_{p(i)}\right|\cG^n\right]&\leq \left(\sum_{i=1}^{\lfloor n^\beta \rfloor} (\hat{D}^-_i)^2\right)^{1/2}\left(\sum_{i=1}^{\lfloor n^\beta \rfloor} \E\left[\left.\hat{D}^+_{p(i)}\right|\cG^n\right]^2\right)^{1/2}\\
&= \left(\sum_{i=1}^{\lfloor n^\beta \rfloor} (\hat{D}^-_i)^2\right)^{1/2}\left(\sum_{j=1}^{\lfloor n^\beta \rfloor} (\hat{D}^+_{j})^2\sum_{i=1}^{\lfloor n^\beta \rfloor}\E\left[\left.\one_{j=p(i)}\right| \cG^n\right]\right)^{1/2}\\ 
&\leq\left(\sum_{i=1}^{\lfloor n^\beta \rfloor} (\hat{D}^-_i)^2\right)^{1/2}\left(\sum_{i=1}^{\lfloor n^\beta \rfloor} (\hat{D}^+_i)^3\right)^{1/2}.\end{align*}
We will show that \begin{equation}\label{eq.conditionalprobanamolousedges}\P\left(\left.B^1_n(\lfloor n^\beta \rfloor)+B^2_n(\lfloor n^\beta \rfloor)+B^3_n(\lfloor n^\beta \rfloor)>0\right| \cG^n \right)\overset{p}{\to}0\end{equation} as $n\to\infty$. We note that 
$$\sum_{i=\lfloor n^\beta \rfloor+1}^n \hat{D}^-_i=\sum_{i=1}^n D^-_i-\sum_{i=1}^{\lfloor n^\beta \rfloor -1}\hat{D}^-_i,$$
and by the weak law of large numbers, $\frac{1}{n}\sum_{i=1}^n D^-_i\overset{p}{\to} \mu n$, so \cref{eq.conditionalprobanamolousedges} follows if we show that 
\begin{enumerate}
    \item $\frac{1}{n}\sum_{i=1}^{\lfloor n^\beta \rfloor} \hat{D}_i^-\overset{p}{\to}0 $, 
    \item $\frac{1}{n}\sum_{i=1}^{\lfloor n^\beta \rfloor}\hat{D}_i^- \hat{D}_i^+\overset{p}{\to}0$,
    \item $\frac{1}{n}\sum_{i=1}^{\lfloor n^\beta \rfloor}(\hat{D}_i^-)^2\overset{p}{\to}0$, and 
    \item $\frac{1}{n}\sum_{i=1}^{\lfloor n^\beta \rfloor}(\hat{D}_i^+)^3\overset{p}{\to}0$
\end{enumerate}
as $n\to \infty$. The proposition will then follow from the bounded convergence theorem.

Note that we can only show the convergence of the Radon-Nikodym derivative $\Phi(n,m)$ under rescaling for $m=O(n^{2/3})$, so it is not straightforward to use the measure change to prove results on the time scale $O(n^\beta)$ for $\beta>2/3$, such as the convergences above. Therefore, instead, we will use \emph{Poissonization} to sample $(\mathbf{\hat{D}}_{n,1},\dots,\mathbf{\hat{D}}_{R_n,n})$. This technique was also used by Joseph in \cite{josephComponentSizesCritical2014}. 

 Let $R_n$ be as before, and, conditional on $R_n$, let $D^{0,+}_1,\dots,D^{0,+}_{n-R_n}$ i.i.d.\ random variables with the law of $D^+$ conditional on the event $\{D^-=0\}$, and set $S_n=\sum_{i=1}^{n-R_n}D^{0,+}_i$. Suppose $R_n=r$ and $S_n=s$. 
Let
$$\pi_0(dt,k_1,k_2)=r\P(D^-=k_1,D^+=k_2|D^->0)k_1\exp(-k_1 t)dt$$
be a measure on $\R_+\times \N^2$, and let $\Pi_0$ be a Poisson point process with intensity measure $\pi_0$ conditional on $\Pi_0(\R,\N,\N)=r$. We view the first coordinate as the time coordinate, and refer to the second and third coordinate as the \emph{point}. Then, the points in $\Pi_0$ ordered by time have the same law as $(\mathbf{\hat{D}}_{n,1},\dots,\mathbf{\hat{D}}_{r,n})$ (before conditioning on the event $\{\sum_{i=1}^nD^-_i=\sum_{i=1}^nD^+_i\}$).  
The intensity of this process is not constant in $t$, so we perform a time change. Define
$$\cL_{\mathbf{D}}(x,y)=\E\left[\left.\exp(-xD^--yD^+)\right|D^->0 \right],$$
and set 
$$\psi(t)=\left(1-\cL_{\mathbf{D}}(\cdot,0)\right)^{-1},$$
so that, by a trivial adaptation of \cite[Lemma 4.1]{josephComponentSizesCritical2014}, for 
$$\pi_r(dt,k_1,k_2):=\P(D^-=k_1,D^+=k_2|D^->0)k_1\exp\left(-k_1 \psi(t/r)\right)\psi'(t/r)dt$$
on $(0,r)\times \N^2$, we have that for $t\in (0,r)$, there exists a probability measure $P_t$ on $\N^2$ such that
$$\pi_r(dt,k_1,k_2)=P_t(D^-=k_1,D^+=k_2)dt.$$
Let ${\Pi}^r$ be a Poisson point process with intensity $\pi_r$. Define $N_r= {\Pi}_r((0,r),\N,\N)$ and $\Delta_r=\int_{(0,r)\times \N^2}(k_1-k_2){\Pi}^r(dt,k_1,k_2)=s$. Then, let ${\Pi}^{r,s}$ have the law of ${\Pi}_r$ conditional on the events $\{N_r=r\}$ and $\{\Delta_r=s\}$. Then, the points of ${\Pi}^{r,s}$ ordered by time are distributed as $(\mathbf{\hat{D}}_{n,1},\dots,\mathbf{\hat{D}}_{n,R_n})$ conditional on the events $\left\{\sum_{i=1}^nD^-_i=\sum_{i=1}^nD^+_i\right\}$, $\{R_n=r\}$ and $\{S_n=s\}$. Let ${\lambda}^{r,s}_t$ be the marginal density of ${\Pi}^{r,s}$ in $t$, so that there exists a probability distribution ${P}^{r,s}_t(k_1,k_2)$ on $\N^2$ such that for ${\pi}^{r,s}_t(k_1,k_2)$ the marginal intensity measure on $\N^2$ of ${\Pi}^{r,s}$ in $t$, 
$${\pi}^{r,s}_t(k_1,k_2)={\lambda}^{r,s}_t{P}^{r,s}_t(k_1,k_2)$$
for all $k_1,k_2\in \N$.

For any $L>0$, define
$$\cE_L=\left\{|R_n-\E[R_n]|\leq Ln^{1/2},  |S_n-\E[S_n]|\leq Ln^{1/2}\right\}.$$
Then, note that 
\begin{align*}\P\left(\frac{1}{n}\sum_{i=1}^{\lfloor n^\beta\rfloor} \hat{D}_i^-\hat{D}_i^+>\epsilon \right)\leq & \P(\cE_L^c) + \P\left(\left.\Pi_{R_n,S_n}\left((0,2n^\beta),\N^2\right)<n^\beta\right|\cE_L \right)\\
&+\P\left(\left.\frac{1}{n}\int_{(0,2n^\beta)\times \N^2}k_1k_2 \Pi_{R_n,S_n}(dt,k_1,k_2)>\epsilon  \right|\cE_L \right)\end{align*}
Fix $\epsilon>0$. By the central limit theorem, we can pick an $L$ such that $\P(\cE_L^c)<\epsilon$ for all $n$. We condition on $\cE_L$. Suppose $R_n=r$ and $S_n=s$. Then, for $P$ a Poisson random variable with rate $2n^\beta$,
 $$\P\left(\Pi_{r,s}\left((0,2n^\beta),\N^2\right)<n^\beta\right)\leq \frac{\P\left(P<n^\beta\right)}{\P(\Delta_r=s, N_r=r)}$$
 We note that the numerator is the probability of a large-deviation event and decreases exponentially fast in $n^\beta$, while the local limit theorem yields that the denominator is of order $n^{-1/2}$ uniformly in all $r$ and $s$ that we consider on $\cE_L$. This implies that $$\P\left(\left.\Pi_{R_n,S_n}\left((0,2n^\beta),\N^2\right)<n^\beta\right|\cE_L\right)\to 0$$
as $n\to \infty$.  
Now, note that for $E^{r,s}_t$ denoting the expectation with respect to $P_t^{r,s}$,
$$\E\left[\frac{1}{n}\int_{(0,2n^\beta)\times \N^2}k_1k_2 \Pi_{r,s}(dt,k_1,k_2) \right]=\frac{1}{n}\int_{(0,2n^\beta)}\lambda^{r,s}_t E^{r,s}_t[D^- D^+]dt,$$
so we start by bounding $E^{r,s}_t[D^- D^+]$. 
We note that
$$E^{r,s}_t\left[{D}^-{D}^+\right]={E}^r_t\left[\left.{D}^-{D}^+\right| \Delta_r=s, N_r=r\right]={E}^r_t\left[{D}^-{D}^+\frac{\P\left[\left. \Delta_n=s, N_r=r \right|\Pi_r(t,D^-,D^+)=1\right]}{\P\left[ \Delta_r=s, N_r=r\right]}\right].$$
By the fact that $\Pi_r$ is a point process, we have that for $k_1$, $k_2$ in $\N$, 
$$\P\left[ \Delta_r=s, N_r=r \left| \Pi_r(t,k_1,k_2)=1\right.\right]=\P\left[ \Delta_r=s+k_2-k_1, N_r=r-1 \right],$$
so that, since $N_r\sim \operatorname{Poisson}(r)$, and since on the event $\{N_r=r-1\}$ (resp. $\{N_r=r\}$),  $\Delta_r-s$ is the sum of $r-1$ (resp. $r$) i.i.d. random variables with finite variance and mean at most $O(n^{-1/2})$, we observe that, by the local limit theorem,
\begin{align*}
    \P\left[ \Delta_r=s, N_r=r \left| \hat{D}^-_t=k_1,\hat{D}^+_t=k_2\right.\right]&=O(n^{-1/2})\text{, and}\\
    \P\left[ \Delta_r=s, N_r=r \right]&=\Theta(n^{-1/2})
\end{align*} for any $k_1$ and $k_2$, and any $r$ and $s$ that we consider on $\cE_L$. Therefore, there exists a $c_1$ such that
$$\frac{\P\left[ \Delta_r=s, N_r=r \left| \hat{D}^-_t=k_1,\hat{D}^+_t=k_2\right.\right]}{\P\left[ \Delta_r=s, N_r=r\right]}<c_1$$
for any $k_1$, $k_2$, $t$ and $n$, and any $r$ and $s$ that we consider on $\cE_L$. If we show that for some $c_2$ $${E}^r_t\left[\hat{D}^-\hat{D}^+\right]<c_2$$ for all $r$ in the interval that we consider and all $t<2n^\beta$, it follows that there is a $c_3$ such that
$${E}^{r,s}_t\left[\hat{D}^-\hat{D}^+\right]<c_3$$ 
for any $k_1$, $k_2$, $t$ and $n$, and any $r$ and $s$ that we consider on $\cE_L$.
We note that by definition of $\pi_r(dt,k_1,k_1)$, 
$${E}^r_t\left[\hat{D}^-\hat{D}^+\right]=\frac{\frac{d^3}{dx^2 dy}\cL_{\mathbf{D}}(x,y)|_{(\psi(t/r),0)}}{\frac{d}{dx}\cL_{\mathbf{D}}(x,y)|_{(\psi(t/r),0)}}.$$
Careful analysis of $\cL_{\mathbf{D}}(x,y)$ and $\psi(s)$ implies that this quantity is bounded uniformly for all $n$, all $r$ in the interval that we consider and all $t\in(0,2n^\beta)$. We refer the reader to the proof of   \cite[Lemma A.1]{josephComponentSizesCritical2014} for the details of a similar argument in the undirected setting.
This implies that 
$$\E\left[\frac{1}{n}\int_{(0,2n^\beta)\times \N^2}k_1k_2 \Pi_{r,s}(dt,k_1,k_2) \right]\leq \frac{C}{n}\E\left[\Pi_{r,s}\left((0,2n^\beta),\N,\N\right)\right].$$
Then, we note that for any $x>0$, for $P$ a Poisson random variable with rate $2n^\beta$,
$$\P\left(\Pi_{r,s}\left((0,2n^\beta),\N,\N\right)>(x+1)2n^\beta\right)\leq \frac{\P\left[P>(x+1)2n^\beta \right]}{\P\left[ \Delta_r=s, N_r=r\right]}.$$
Then, by the local limit theorem and the exponential tail of the Poisson distribution, we obtain that there exist $c_4,c_5>0$ such that for all $n$, all $r$ and $s$ in the interval of interest and all $x>1$,
$$\P\left(\Pi_{r,s}\left((0,2n^\beta),\N,\N\right)>(x+1)2n^\beta\right)\leq c_4\exp(-c_5xn^\beta).$$
This implies that there is a constant $c_6$ such that 
$$\E\left[\Pi_{r,s}\left((0,2n^\beta),\N,\N\right)\right]\leq c_6 n^\beta$$
for all $n$ and all $r$ and $s$ that we consider under $\cE_L$. 
It then follows that 
$$\E\left[\frac{1}{n}\int_{(0,2n^\beta)\times \N^2}k_1k_2 \Pi_{r,s}(dt,k_1,k_2) \right]\to 0$$
as $n\to \infty$ uniformly in all $r$ and $s$ of interest, so for $n$ large enough,
$$\P\left(\left.\frac{1}{n}\int_{(0,2n^\beta)\times \N^2}k_1k_2 \Pi_{R_n,S_n}(dt,k_1,k_2)>\epsilon  \right|\cE_L \right)<\epsilon.$$
This implies that
$$\frac{1}{n}\sum_{i=1}^{\lfloor n^\beta \rfloor}\hat{D}_i^- \hat{D}_i^+\overset{p}{\to}0.$$
The other convergences are proved similarly, and the result follows. 
\end{proof}