\section{Convergence of the SCCs under rescaling}\label{sec.convSCCs}

In this section, we will use the convergence of the out-forest that we obtained in Section \ref{sec.convoutforest} to show that the SCCs ordered by decreasing length converge under rescaling in the $d_{\vec{\cG}}$-product topology. The structure of the argument is as follows. 
% \begin{itemize}
%     \item In Subsection XXX, we categorise the different type of surplus edges, and identify which of them are part of an SCC. We call these \emph{important} surplus edges.
%     \item A key category of \emph{important} surplus edges is the \emph{ancestral surplus edges}. Namely, a component of $\hat{F}_n(k)$ can not contain a non-trivial SCC if it does not contain an ancestral surplus edge. In Subsection XXX, we use this fact to identify the components of $\hat{F}_n\left(\lfloor n^{2/3}T\rfloor \right)$ that contain a non-trivial SCC, and show that their metric structure converges jointly with the position of their first ancestral surplus edge. 
%     \item In Subsection XXX, we define a procedure that, given a component of $\hat{F}_n\left(\lfloor n^{2/3}T\rfloor \right)$ and its first ancestral surplus edge, identifies a finite set of edges that contains all important surplus edges. We show that this set converges under rescaling. 
%     \item In Subsection XXX, we define the vertex identifications and the cutting procedure that we use to extract the non-trivial SCCs from the components of $\hat{F}_n\left(\lfloor n^{2/3}T\rfloor \right)$ and the positions of the important surplus edges. We show that this cutting procedure converges.
%     \item In Subsection XXX, we show that for a given $\delta>0$, the SCCs with length at least $\delta$ are contained in $\hat{F}_n\left(\lfloor n^{2/3}T\rfloor \right)$ for $T$ large enough with high probability. This implies that convergence under rescaling of the SCCs contained in $\hat{F}_n\left(m_n \right)$ for any $m_n=O(n^{2/3})$ is sufficient to obtain convergence of the SCCs ordered by length in the $d_G$-product topology. 
% \end{itemize}


\subsection{Convergence of the out-components that contain an ancestral surplus edge}\label{subsec.ancestral}
In this subsection, we will prove that the components of $\hat{F}_n\left(\lfloor n^{2/3}t\rfloor\right)$ that contain an ancestral surplus edge converge under rescaling. Recall the definition of $(A_n(k),k\geq 1)$ from Subsection \ref{subsubsec.samplecandidates}, and recall that the components in $(\hat{F}_n(k),k\geq 1)$ that contain a non-trivial SCC correspond to the components in $(\hat{F}_n(k),k\geq 1)$ on which $(A_n(k),k\geq 1)$ increases. Moreover, if $(A_n(k),k\geq 1)$ increases on a component, the law of the first increase time corresponds to the position of the tail of the first ancestral surplus edge in the component. \\
We first study the convergence of $(\hat{H}_n^\ell(k),k\geq 1)$ under rescaling. This is an extension of Theorem \ref{thm.convoutforest}.
\begin{lemma}\label{lemma.heightprocesswithlengths}
Let $(B_t, t\geq 0)$ be a Brownian motion, and define
$$(\hat{B}_t,t\geq 0):=\left( B_t-\frac{\sigma_{-+}+\nu_-}{2\sigma_+ \mu}t^2, t\geq 0\right),$$ and $$(\hat{R}_t,t\geq 0)=\left(\hat{B}_t-\inf\left\{\hat{B}_s: s\leq t\right\},t\geq 0\right).$$ 
Then,
\begin{align*}&\left(n^{-1/3}\hat{S}^{+}_n\left(\lfloor n^{2/3}t\rfloor \right),n^{-1/3}\hat{H}_{n}\left(\lfloor n^{2/3}t\rfloor \right),n^{-1/3}\hat{H}^\ell_{n}\left(\lfloor n^{2/3}t\rfloor \right),  t\geq 0\right)\\
&\todist\left(\sigma_+ \hat{B}_t, \frac{2}{\sigma_+} \hat{R}_t,\frac{2(\sigma_{-+}+\nu_-)}{\sigma_+\mu} \hat{R}_t, t\geq 0\right)\end{align*}
in $\D(\R_+,\R)^3$,
jointly with 
$$\left(n^{-2/3}\hat{S}_n^-\left(\lfloor n^{2/3}t\rfloor \right), n^{-1/3}\hat{P}_n\left(\lfloor n^{2/3}t\rfloor \right),t\geq 0\right)\overset{p}{\to}\left(\nu_-t,  \frac{\nu_-}{2\mu} t^2, t\geq 0\right)$$
in $\D(\R_+,\R)^2$ as $n\to \infty$.
\end{lemma}
\begin{proof}
We use Theorem 1 in \cite{deraphelisScalingLimitMultitype2017} by de Raphélis, that shows convergence of the height process of a Galton-Watson forest with edge-lengths under a few conditions on the degree and edge length distribution. We will apply this result to the black-purple-red Galton-Watson forest $(F^{pr}(k),k\geq 1)$, as defined in Subsection \ref{subsubsec.convheightprocess}. \\
We equip $(F^{pr}(k),k\geq 1)$ with edge lengths in the following manner. For a purple or red vertex with out degree $d^+$, sample its in-degree with law $Z^-$ conditioned on $Z^+=d^+$. The in-degree of the black vertices is encoded by $(Y^{-}(k),k\geq 1)$.  Then, for a vertex with in-degree $d^-$, let the edges connecting it to its children have length $d^--1$ (unless it is the root of the component, then let the edges connecting to its children have length $d^-$). Call the resulting forest with edge lengths $(F^{pr,\ell}(k),k\geq 1)$, and let $(H^{pr,\ell}(k),k\geq 1)$ be the corresponding height process.\\
We will translate the conditions of Theorem 1 in \cite{deraphelisScalingLimitMultitype2017} to our setting and check them. The conditions are as follows.
\begin{enumerate}
    \item $\E[Z^+]=1$
    \item $1<\E[(Z^+)^2]<\infty$
    \item $\E\left[Z^+\one_{\{Z^->x\}}\right]=o(x^{-2})$ as $x\to \infty$
\end{enumerate}
Under these conditions, using the notation from Subsection \ref{subsubsec.convheightprocess},
\begin{align}\begin{split}\label{eq.convmodifiedheightprocess}
&\left(n^{-1/3}Y^{pr}\left(\lfloor t n^{2/3}\rfloor \right),n^{-1/3}H^{pr}\left(\lfloor t n^{2/3}\rfloor \right), n^{-1/3}H^{pr,\ell}\left(\lfloor t n^{2/3}\rfloor \right),t\geq 0\right)\\
&\todist\left(\sigma_+ B_s, \frac{2}{\sigma_+}R_s, \frac{2(\sigma_{+-}+\nu_-)}{\mu\sigma_+}R_s,t\geq 0\right)
\end{split}\end{align}
in $D(\R_+,\R)^3$ as $n\to \infty$. 
Then, we observe that the rest of the argument in Subsection \ref{subsubsec.convheightprocess} Subsection \ref{subsubsec.convaftermeasurechange}, and Appendix \ref{appendix.heightprocessblackpurplered} can be extended to include the height process with edge lengths. This yields the result.\\
Therefore, to finish the proof, we need the conditions of Theorem 1 in \cite{deraphelisScalingLimitMultitype2017} to hold. The conditions are equivalent to 
\begin{enumerate}
    \item $\E[D^+D^-]=\E[D^-]$
    \item $1<\frac{\E[(D^+)^2D^-]}{\E[D^-]}<\infty$
    \item $\E\left[D^+D^-\one_{D^->x}\right]=o(x^{-2})$ as $x\to \infty$. 
\end{enumerate}
Note that the first and second condition follow directly from the assumptions, and the third condition is implied by $\E[D^+(D^-)^3]<\infty$.
\end{proof}



\begin{proposition}\label{prop.convergenceancestraledges}
We have, jointly with the convergence in Lemma \ref{lemma.heightprocesswithlengths},
\begin{align*}\left(A_n\left(\lfloor tn^{2/3}\rfloor\right),t\geq 0\right)\todist\left(A_t,t\geq 0\right),\end{align*}
as $n\to \infty$, where $(A_t,t\geq 0)$ is a Cox process of intensity $$\frac{2(\sigma_{-+}+\nu_-)}{\sigma_+\mu^2} \hat{R}_t$$ at time $t$. The convergence is in $D(\R_+,\R)$.
\end{proposition}


\begin{proof}
By definition, $(A_n(k),k\geq 1)$ is a counting process with compensator 
\begin{align*}
    A_n^{comp}(k)&=\sum_{i=1}^k \frac{\hat{H}^\ell(i)}{\hat{S}^-_n(i)}\one_{\{\hat{P}_n(i)-\hat{P}_n(i-1)=1\}}\\
    &=\sum_{j=1}^{\hat{P}_n(k)}\frac{\hat{H}^\ell(\min\{l:\hat{P}_n(l)\geq k\})}{\hat{S}^-_n(\min\{l:\hat{P}_n(l)\geq k\})},
\end{align*}
 By Theorem 14.2.VIII of Daley and Vere-Jones \cite{daleyIntroductionTheoryPoint2008}, the claimed convergence under rescaling of $(A_n(k),k\geq 1)$ follows if we show that 
\begin{equation}\label{eq.convergencecompensator}
    \left(A_n^{comp}\left(\lfloor tn^{2/3}\rfloor \right), t\geq 0\right)\todist\left(\frac{2(\sigma_{-+}+\nu_-)}{\sigma_+\mu^2} \int_0^t\hat{R}_v dv, t \geq 0\right)
\end{equation}
in $\D(\R_+,\R)$ as $n\to \infty$ jointly with the convergence in Lemma \ref{lemma.heightprocesswithlengths}. Therefore, we will now prove that \eqref{eq.convergencecompensator} holds. 
By
$$\left(n^{-1/3}\hat{P}_n\left(\lfloor n^{2/3}t\rfloor \right),t\geq 0\right)\overset{p}{\to}\left(\frac{\nu_-}{2\mu}t^2,t\geq 0\right)$$
in $\D(\R_+,\R)$ as $n\to \infty$,
we get that
\begin{align*}\left(n^{-2/3}\min\{l\geq 1:n^{-1/3}\hat{P}_n(l)\geq t\},t\geq 0\right)&\overset{p}{\to}\left(\min\left\{s>0: \frac{\nu_-}{2\mu}s^2\geq t\right \}, t\geq 0\right)\\
&=:\left(p^{-1}(t),t\geq 0\right) \end{align*}
in $\D(\R_+,\R)$ as $n\to \infty$, because $\left(\frac{\nu_-}{2\mu}t^2,t\geq 0\right)$ is strictly increasing. Then, Lemma \ref{lemma.heightprocesswithlengths}, Lemma \ref{lemma.technicalcomposedfunctions}, Slutsky's lemma and the continuous mapping theorem imply that 
\begin{align*}&\left(\sum_{j=1}^{\lfloor n^{1/3}t\rfloor}\frac{\hat{H}^\ell(\min\{l:\hat{P}_n(l)\geq k\})}{\hat{S}^-_n(\min\{l:\hat{P}_n(l)\geq k\})},t\geq 0\right)\todist \left( \frac{2(\sigma_{-+}+\nu_-)}{\sigma_+\mu} \int_0^t \frac{\hat{R}_{p^{-1}(s)}}{\nu_- p^{-1}(s)}ds,t\geq 0 \right)
\end{align*}
in $D(\R_+,\R)$ as $n\to\infty$. If we combine this with the convergence under rescaling of $(P_n(k),k\geq 1)$ and apply Lemma \ref{lemma.technicalcomposedfunctions}, some simple analysis then yields \eqref{eq.convergencecompensator}, which proves the statement.
\end{proof}

\subsection{Finding the important components in the out-forest}\label{subsec.componentswithancestral}
In this subsection, we will show that, conditional on the convergence under rescaling in Proposition \ref{prop.convergenceancestraledges}, the sequence of intervals that encode the trees with ancestral surplus edges in $(\hat{F}_n(k),k\leq \lfloor T n^{2/3}\rfloor )$ converges as well under rescaling. Lemma \ref{lemma.extractexcursions} is a statement about extracting excursion intervals from deterministic functions with marks, which we will apply to the sample paths of $(\hat{S}_n^{+}(k),k\geq 1)$ and the increase times of $(A_n(k),k\geq 1)$. The lemma tells us that if the sample paths and increase times converge under rescaling, the beginnings and endpoints of the excursions above the running infimum that contain the increase times converge as well. 
\begin{lemma}\label{lemma.extractexcursions}
Let $(f_n(t), t\geq 0)$ for $n\geq 1$, and $(f(t),t\geq 0)$ be functions in $\D(\R_+,\R)$, such that 
$$(f_n(t), t\geq 0)\to (f(t),t\geq 0)$$ in $\D(\R_+,\R)$ as $n\to \infty$. Assume that $(f(t),t\geq 0)$ is continuous, that $f(t)\to -\infty$ as $t\to \infty$, and that the local minima of $(f(t),t\geq 0)$ are unique. Moreover, let $(x_i^n)_{1\leq i\leq m}$, for $n\geq 1$, and $(x_i)_{1\leq i\leq m}$ be elements of $\R^{m}$ such that for all $i\in [m]$, $x_i^n\to x_i$ in $\R$ as $n\to \infty$, and such that $f(x_i)-\inf\{f(s):s\leq x_i\}>0$ for all $i\in [m]$. Define
\begin{align*}l_i^n&=\inf\left\{t\geq 0:f_n(t)=\inf\{f_n(s):s\leq x_i^n\}\right\}\text{ for }i\in [m]\text{, }n\geq 1\\
\sigma_i^n&=\inf\left\{ t\geq 0: \inf\{f_n(s):s\leq l_i^n+t\} < \inf\{f_n(s):s\leq x_i^n\}\right\}\text{ for }i\in [m]\text{, }n\geq 1\\
l_i&=\inf\left\{t\geq 0:f(t)=\inf\{f(s):s\leq x_i\}\right\},\text{ and}\\
\sigma_i&=\inf\left\{ t\geq 0: \inf\{f(s):s\leq l_i+ t\} < \inf\{f(s):s\leq x_i\}\right\}.
\end{align*}
For $S=\{(a_i,b_i), i\in [m]\}$, let $\operatorname{ord}(S)$ be a sequence consisting of the elements of $S$ put in increasing order of $a_i$, with ties broken arbitrarily, and concatenated with $(0,0)_{i\geq 1}$ so that $\operatorname{ord}(S)\in (\R^2)^\infty$. Then, 
$$\operatorname{ord}\left(\left\{(l_i^n,\sigma_i^n):1\leq i \leq m\right\}\right)\to \operatorname{ord}\left(\left\{(l_i,\sigma_i):1\leq i \leq m\right\}\right)$$
in $(\R^{2})^\infty$ in the $\ell_1$-topology as $n\to \infty$. 
\end{lemma}
\begin{proof}
First, note that $l_i^n$, $\sigma_i^n$, $l_i$, and $\sigma_i$ are well-defined for all $i\in [m]$, $n\geq 1$ by $f(t)\to -\infty$ as $t\to \infty$ and the convergence of $f_n$ to $f$. \\
Fix $i$. We will first show that $l^n_i\to l_i$ and $\sigma_i^n\to \sigma_i$ as $n\to \infty$. Firstly, note that by the assumption that $f(x_i)-\inf\{f(s):s\leq x_i\}>0$ and the continuity of $f$, $l_i<x_i<l_i+\sigma_i$. Fix $0<\epsilon<\min\{x_i-l_i,l_i+\sigma_i-x_i\}/2$. We claim that the following conditions are sufficient for $l^n_i\to l_i$ and $\sigma_i^n\to \sigma_i$ as $n\to \infty$. For all $n$ large enough,
\begin{enumerate}
    \item \label{cond.excursions1} $l_i+\epsilon<x^n_i<l_i+\sigma_i-\epsilon$
    \item \label{cond.excursions2}$\inf\left\{f_n(s):s\in (l_i-\epsilon, l_i+\epsilon)\right\}<\inf\left\{f_n(s):s\in [l_i+\epsilon,l_i+\sigma_i-\epsilon] \right\}$, 
    \item \label{cond.excursions3}$\inf\left\{f_n(s):s\in (l_i-\epsilon, l_i+\epsilon)\right\}<\inf\left\{f_n(s):s\in [0,l_i-\epsilon]\right\}$,
    \item \label{cond.excursions4} $\inf\left\{ f_n(s):s\in (l_i+\sigma_i-\epsilon,l_i+\sigma_i+\epsilon)\right\}<\inf\left\{f_n(s):s\in [0,l_i+\sigma_i-\epsilon]\right\}$
\end{enumerate}
Indeed, condition \ref{cond.excursions1}, \ref{cond.excursions2} and \ref{cond.excursions3} imply $|l^n_i-l_i|<\epsilon$, while condition \ref{cond.excursions1}, \ref{cond.excursions2} and \ref{cond.excursions4} imply $|(l^n_i+\sigma^n_i)-(l_i+\sigma_i)|<\epsilon$. Note that the condition \ref{cond.excursions1} holds for $n$ large enough by definition of $\epsilon$ and the convergence of $x_i^n$ to $x_i$. To show the other conditions, define
\begin{align*}\delta_1&=\inf\left\{f(s):s\in [l_i+\epsilon,l_i+\sigma_i-\epsilon]\right\}-\inf\left\{f(s):s\in (l_i-\epsilon,l_i+\epsilon)\right\}\\
\delta_2&=\inf\left\{f(s):s\in [0,l_i-\epsilon]\right\}-\inf\left\{f(s):s\in (l_i-\epsilon,l_i+\epsilon)\right\}\\
\delta_3&=\inf\left\{f(s):s\in [0,l_i+\sigma_i-\epsilon]\right\}-\inf\left\{f(s):s\in (l_i+\sigma_i-\epsilon,l_i+\sigma_i+\epsilon)\right\}
\end{align*}
By uniqueness of local minima and the definition of $l_i$ and $\sigma_i$,  $\delta:=\min\{\delta_1,\delta_2,\delta_3\}/3>0$. Then, note that for $n$ large enough, $\sup\{|f_n(s)-f(s)|:s\leq l_i+\epsilon\}<\delta$, which implies conditions \ref{cond.excursions2}, \ref{cond.excursions3}, and \ref{cond.excursions4} for such $n$. \\
Since $i$ was arbitrary, and $m$ is finite, we find that $$(l_i^n,\sigma_i^n)_{1\leq i\leq m}\to (l_i,\sigma_i)_{1\leq i\leq m}$$
in $\R^{2m}$ as $n\to \infty$. \\
We now claim that $l_i^n\to l_i$ and $l_j^n\to l_i$ implies that $l_i^n=l_j^n$ for $n$ large enough. Indeed, by definition of $l_i^n$, $l_j^n$ and $\sigma_i^n$, we have that $l_i^n<l_j^n$ implies that $l_j^n-l_i^n\geq \sigma_i^n$, and by the argument above, $\sigma_i^n\to \sigma_i>0$, so $l_i^n-l_j^n\to 0$ can only hold if $l_i^n=l_j^n$ for $n$ large enough. This implies that 
$$\#\left\{(l_i^n,\sigma_i^n):1\leq i \leq m\right\}\to \#\left\{(l_i,\sigma_i):1\leq i \leq m\right\}.$$
Then, the result follows.
\end{proof}

We now apply this result to our process to extract the excursion intervals that contain the marks representing ancestral backedges.
\begin{proposition}\label{prop.extractexcursions}
Fix $T>0$. Use notation as before. For $i\in \left[A_n\left(\lfloor T n^{2/3}\rfloor\right)\right]$, set $X_i^n=\min\{k:A_n(k)=i\}$. Similarly, for $i$ in $\left[A_T\right]$, set $X_i=\min\{t:A_T=i\}$. Define, for $n\geq 1$,
\begin{align*}L_i^n&=\min\left\{k\geq 1:\hat{S}^{p,+}_n(k)=\min\{\hat{S}^{p,+}_n(l):l\leq X_i^n\}\right\}\text{ for }i\in \left[A_n\left(\lfloor T n^{2/3}\rfloor\right)\right]\\
\Sigma_i^n&=\min\left\{k \geq 1: \min\left\{\hat{S}^{p,+}_n(l):l\leq L_i^n+k\right\} < \min\left\{\hat{S}^{p,+}_n(l):l\leq X_i^n\right\}\right\}\text{ for }i\in \left[A_n\left(\lfloor T n^{2/3}\rfloor\right)\right]\\
L_i&=\inf\left\{t\geq 0:\sigma_+\hat{B}_t=\inf\{\sigma_+\hat{B}_s:s\leq X_i\}\right\}\text{ for }i\in \left[A(T )\right]\text{ and}\\
\Sigma_i&=\inf\left\{ t\geq 0: \inf\{\sigma_+\hat{B}_s:s\leq L_i+t\} < \inf\{\sigma_+\hat{B}_s:s\leq X_i\}\right\}\text{ for }i\in \left[A(T )\right].
\end{align*}
Then, for $\operatorname{ord}$ defined as in the statement of Lemma \ref{lemma.extractexcursions}, we get that
$$\operatorname{ord}\left(\left\{\left(n^{-2/3}L_i^n,n^{-2/3}\Sigma_i^n\right):1\leq i \leq A_n\left(\lfloor T n^{2/3}\rfloor\right)\right\}\right)\todist \operatorname{ord}\left(\left\{(L_i,\Sigma_i):1\leq i \leq A_T\right\}\right)$$
in the $\ell_1$-topology on $(\R^2)^\infty$ as $n\to \infty$, jointly with the convergence in Proposition \ref{prop.convergenceancestraledges}. 
\end{proposition}
\begin{proof}
By Skorokhod's representation theorem, we may work on a probability space where the convergence in Proposition \ref{prop.convergenceancestraledges} holds almost surely, and claim that we can apply Lemma \ref{lemma.extractexcursions} to the sample paths of $\left(n^{-1/3}\hat{S}^{p}_n\left(\lfloor n^{2/3}t\rfloor\right),t \geq 0\right)$ with marks $$\left(n^{-2/3}X_n^i\right)_{1\leq i\leq A_n\left(\lfloor T n^{2/3}\rfloor\right)}.$$ We check the conditions.
Firstly, note that by $A_n\left(\lfloor T n^{2/3}\rfloor\right)\to A\left(T\right)$ almost surely as $n\to \infty$, we can pick $n$ large enough such that $A_n\left(\lfloor T n^{2/3}\rfloor\right)=A\left(T\right)$, where we ignore events of $0$ probability. Furthermore, we observe that $(\hat{B}_t,t\geq 0)$ is a Brownian motion with negative parabolic drift, so the sample paths of $(\sigma_+\hat{B}_t,t\geq 0)$ are continuous and drift to $-\infty$ almost surely. By the local absolute continuity of $(\hat{B}_t,t\geq 0)$ to a Brownian motion, its local minima are almost surely unique. By 
$$\left(A_n\left(\lfloor t n^{2/3}\rfloor\right), t\leq T\right) \overset{a.s.}{\to}\left(A\left(t\right),t\geq 0\right)$$
in $\D(\R_+,\R)$ as $n\to \infty$, we observe that for all $i\in [A_T]$, $n^{-2/3}X_i^n\to X_i$ almost surely in $\R$ as $n\to \infty$. The intensity of $(A_t,t\geq 0)$ at time $t$ is proportional to $\hat{R}_t$, so $\hat{R}_{X_i}>0$ for all $i$ almost surely. This allows us to apply Lemma \ref{lemma.extractexcursions}, and the convergence follows.
\end{proof}

% Given the convergence of the excursion intervals that contain the ancestral surplus edges, it is straightforward to obtain convergence of the encoding processes of the components of $(\hat{F}_n(k), k\geq 1)$ that contain the marks. 
% \begin{corollary}\label{cor.convergencesequenceofcomponents}
% Recall notation in the statement of Proposition \ref{prop.extractexcursions}. For $i\in \left[A_n\left(\lfloor T n^{2/3}\rfloor\right)\right]$, set
% \begin{align*}
%     &\left(\mathsf{h}_i^n,\mathsf{h}_i^{\ell,n}, \mathsf{s}_i^{-,n}, \mathsf{p}^n_i, \mathsf{a}_i^n\right)\\
%     &=\left(n^{-1/3}\hat{H}_n\left(L_i^n+\lfloor n^{2/3} t \rfloor\right),n^{-1/3}\hat{H}^{\ell}_n\left(L_i^n+\lfloor n^{2/3} t \rfloor\right),  n^{-2/3}\hat{S}^{p,-}_n\left(L_i^n+\lfloor n^{2/3} t \rfloor\right),\right.\\
%      &\qquad \left.n^{-1/3}\left(\hat{P}_n\left(L_i^n+\lfloor n^{2/3} t \rfloor\right)-\hat{P}_n\left(L_i^n\right)\right), \hat{A}_n\left(L_i^n+\lfloor n^{2/3} t \rfloor\right), 0\leq t \leq n^{-2/3}\Sigma_i^n\right),
% \end{align*}
% such that $\left(\mathsf{h}_i^n,\mathsf{h}_i^{\ell,n}, \mathsf{s}_i^{-,n}, \mathsf{p}^n_i, \mathsf{a}_i^n\right)$ encodes the component of $(\hat{F}_n(k), k\geq 1)$ that contains the $i^{th}$ increase time of $(A_n(k),k\geq 1)$.
% Similarly, for $i\in \left[A_T\right]$, set
% \begin{align*}
%     &\left(\mathsf{h}_i,\mathsf{h}_i^{\ell}, \mathsf{s}_i^{-}, \mathsf{p}_i, \mathsf{a}_i\right)\\
%     &=\left(\frac{2}{\sigma_+}\hat{R}_{L_i+t},\frac{2(\sigma_{-+}+\nu_-)}{\sigma_+\mu}\hat{R}_{L_i+t},  \nu_-(L_i+t),\frac{\nu_-}{2\mu}\left((L_i+t)^2-L_i^2\right),A(L_i+t), 0\leq t \leq \Sigma_i\right).
% \end{align*}
% For $S=\{(a_i,\mathbf{b}_i), i\in [m]\}$, let $\operatorname{ord}(S)$ be a sequence consisting of the elements of $S$ put in decreasing order of $a_i$, with ties broken arbitrarily, and concatenated with $(0,\mathbf{0})_{i\geq 1}$.\\
% Then, 
% \begin{align*}&\operatorname{ord}\left(\left\{\Sigma_i^n, \left(\mathsf{h}_i^n,\mathsf{h}_i^{\ell,n}, \mathsf{s}_i^{-,n}, \mathsf{p}^n_i, \mathsf{a}_i^n\right), i\in \left[A_n\left(\lfloor T n^{2/3}\rfloor\right)\right]\right\}\right)\\
% &\todist\operatorname{ord}\left(\left\{\Sigma_i, \left(\mathsf{h}_i,\mathsf{h}_i^{\ell}, \mathsf{s}_i^{-}, \mathsf{p}_i, \mathsf{a}_i, \right), i\in \left[A_T\right]\right\}\right)\end{align*}
% as $n\to \infty$. The topology on càdlàg functions of the form $(f(t),t\leq a)$ is given by embedding the functions in $D(\R_+,\R)$ by considering $(\tilde{f}(t),t\geq 0)=(f(t)\one_{t<a}, t\geq 0)$. Then, we use the $\ell_1$-topology on the sequence space.
% \end{corollary}
% \begin{proof}
% This is a consequence of Proposition \ref{prop.convergenceancestraledges} and Proposition \ref{prop.extractexcursions}. 
% \end{proof}

\subsection{Convergence of the set of candidates}
By Proposition \ref{prop.extractexcursions}, we know that the intervals that encode the out-components that contain an ancestral surplus edge converge under rescaling. This convergence holds jointly with the convergence under rescaling of the first time step at which an ancestral surplus is found in each of these components. We will show that the positions of the other candidates in a component converge as well under rescaling. Recall the procedure to sample candidates that is described in Subsection \ref{lemma.samplecandidates}. 
% \begin{lemma}\label{lemma.samplingprocedure}
% Suppose we are exploring the component of $(\hat{F}_n(k),k\geq 1)$ that contains vertex $k$, and suppose $k$ is purple. Denote this component by $\cT$, with root $l$. Moreover, suppose the tails of all important surplus edges in $\cT$ that are discovered up to time $k$ are contained in $C_k\subset\{g+1,\dots,k-1\}$. Then, for $S$ a subset of the vertices of $\cT$, let $\cT(S)$ be the subtree of $\cT$ spanned by $S$. Then, $k$ is the tail of an important surplus edge only if the surplus edge corresponding to $k$ has its head in $\cT(C_k\cup\{g,k
% \})$. 
% \end{lemma}
% \begin{proof}
% This is a direct consequence of Lemma \ref{lemma.whatispartofscc}.\ref{item.factsonsccs2} and \ref{lemma.whatispartofscc}.\ref{item.factsonsccs4}. 
% \end{proof}


The following proposition shows convergence under rescaling of the sequence of tails of the candidates on a particular component of $(\hat{F}_n(k),k\geq 1)$. 
\begin{proposition}\label{prop.convergencestartingpointscandidates}
Fix $T>0$. By Skorokhod's representation theorem, we may work on a probability space where the convergence in Propositions \ref{prop.convergenceancestraledges} and \ref{prop.extractexcursions} holds almost surely. Let $(l,\sigma)\in \left\{(L_i,\Sigma_i):i\leq A_T\right\}$, so that, for each $n$ large enough, we can find a $(l_n,\sigma_n)\in\left\{(L_i^n,\Sigma_i^n):i\leq A_n\left(\lfloor Tn^{2/3}\rfloor\right)\right\}$, such that $(l_n,\sigma_n)\to (l,\sigma)$. Set $V_1=\inf\{t\in [l,l+\sigma]:A(t)=A(l)+1\}$, and similary, set $V_1^n=\min\{l_n<k\leq l_n+\sigma_n:A_n(k)=A_n(l_n)+1\}$, which are well-defined by definition of $l$, $\sigma$, $l_n$ and $\sigma_n$.  Then, by construction, $\{l_n+1,\dots,l_n+\sigma_n\}$ encodes a component of $(\hat{F}_n(k),k\geq 1)$. Call this component $T^n_{l_n}$. We apply the procedure defined in \ref{lemma.samplecandidates} to find the tail of candidates in $T^n_{l_n}$. Let $\mathbf{C}_n(l_n)$ denote the sequence of tails of candidates in $T^n_{l_n}$. Similarly, $[l,l+\sigma]$ encodes a component of $(\hat{\cF}(t),t\geq 0)$. Call this component $\cT_l$, and apply procedure in Subsection \ref{subsubsec.samplecontinuousobject} to find the tails of candidates in $\cT_l$, and denote its sequence of tails of candidates by $\mathbf{V}(l)$. Then, jointly with the convergence in Proposition \ref{prop.extractexcursions}, 
$$n^{-2/3}\mathbf{V}_n(l_n)\todist\mathbf{V}(l)$$
in the $\ell_1$ topology.
\end{proposition}
\begin{proof}
We will find a coupling such that $n^{-2/3}\mathbf{V}_n(l_n)\overset{a.s.}{\to}\mathbf{V}(l).$ By the convergence in Propositions \ref{prop.convergenceancestraledges} and \ref{prop.extractexcursions}, $n^{-2/3}V_1^n\overset{a.s.}{\to}V_1$. In general, let $V_m^n$ denote the $m^{th}$ candidate that is found in $T^n_{l_n}$, and let $V_m$ denote the $m^{th}$ candidate that is found in $\cT_l$. Suppose that, for some $m$, we have found a coupling such that 
\begin{equation}\label{eq.convergencemcandidates}n^{-2/3}(V_1^n,\dots,V_m^n)\overset{a.s.}{\to}(V_1,\dots,V_m).\end{equation}
Then, $V_{m+1}^n$ is distributed as the position of the first jump of a counting process $K^n_{m+1}(k)$ on $[0,\infty)$ with compensator 
$$K^n_{comp,m+1}(k)=\sum_{i=V_m^n+1}^k \frac{\ell\left(T^{n,\text{mk}}_{i}\right)-m}{\hat{S}^-(i)}  \one{\left\{P_n(i)=P_n(i-1)+1\right\}}$$
for $k\in [V_m^n+1,l_n+\sigma_n]$ and $0$ otherwise, where $T^{n,\text{mk}}_{i}$ is the subtree of $T^n_{l_n}$ spanned by $\{l_n+1,V^n_1,\dots,V^n_m,i\}$. 
Moreover, for $T_s$ the subtree of $\cT_l$ spanned by $\{l,V_1,\dots,V_m, s\}$, and $|T_s|$ its length as encoded by $\left(\frac{2}{\sigma_+}\hat{R}_t,t\geq 0\right)$, $V_{m+1}$ is the first jump in a counting process $K_{m+1}(t)$ on $[0,\infty)$ with compensator 
$$K_{comp,m+1}(t)= \int_{V_m}^t\frac{\sigma_{-+}+\nu_-}{\mu^2}|T_s|ds$$
for $t\in [V_m,L+\Sigma]$ and $0$ otherwise. By the convergence under rescaling of $(\hat{H}^\ell_n(k),k\geq 1)$ in Lemma \ref{lemma.heightprocesswithlengths}, and by Proposition \ref{prop.extractexcursions}, we get that the metric structure of $T^n_{l_n}$ with distances defined by $(\hat{H}^\ell_n(k),k\geq 1)$, and its projection onto $[n^{-2/3}(l_n+1),n^{-2/3}(l_n+\sigma_n)]$, converge under rescaling to the metric structure of $\cT_l$ with distances defined by $$\left(\frac{2(\sigma_{-+}+\nu_-)}{\sigma_+\mu}\hat{R}_t,t\geq 0\right)$$ and its projection onto $[l,\sigma]$. This, combined with \eqref{eq.convergencemcandidates} implies that 
$$\left(n^{-1/3}\ell\left(T^{n,\text{mk}}_{\lfloor t n^{2/3}\rfloor}\right),V_m\leq t \leq l+\sigma\right)\overset{a.s.}{\to} \left(\frac{\sigma_{-+}+\nu_-}{\mu^2}|T^{\text{mk}}_t|, V_m\leq t \leq l+\sigma\right)$$ in $\D([V_m,l+\sigma],\R_+)$ as $n\to \infty$. Then, a similar argument as used in the proof of Proposition \ref{prop.convergenceancestraledges} implies that 
$$\left(K^n_{comp,m+1}\left(\lfloor t n^{2/3}\rfloor \right),V_m\leq t \leq l+\sigma\right)\overset{a.s.}{\to}\left(K_{comp,m+1}(t),V_m\leq t \leq l+\sigma\right),$$
$\D(\R_+,\R_+)$ as $n\to\infty$, which implies that 
$$(K^n_{m+1}(\lfloor t n^{2/3} \rfloor ),t\geq 0)\todist (K_{m+1}(t),t\geq 0)$$ in $\D(\R_+,\R_+)$ as $n\to\infty$, and in particular, we can find a coupling such that $K_m(\infty)>0$ if and only only if $K^n_m(\infty)>0$ for all $n$ large enough, and such that on this event,
$$n^{-2/3}V_{m+1}^n\overset{a.s.}{\to}V_{m+1}.$$
If $K_m(\infty)=0$, set $\mathbf{V}(l)=(V_1,\dots,V_m)$, $\mathbf{V}_n(l_n)=(V^n_1,\dots,V^n_m)$, and the statement follows. If $K_m(\infty)>0$, apply the induction step to $(V_1,\dots,V_{m+1})$ and $(V^n_1,\dots,V^n_{m+1})$. The fact that $|\mathbf{V}(l)|<\infty$ almost surely, as shown in Subsection \ref{subsubsec.samplecontinuousobject}, implies that the induction terminates.
\end{proof}

The following proposition shows that the law of the heads of the candidates converges as well under rescaling, and that the convergence holds in the pointed Gromov-Hausdorff topology. 
\begin{proposition}\label{prop.convergenceheadscandidates}
Suppose the convergence in Propositions \ref{prop.convergenceancestraledges}, \ref{prop.extractexcursions} and \ref{prop.convergencestartingpointscandidates} holds almost surely. Then, for $\mathbf{V}_n(l_n)=(V^n_1,\dots, V^n_{N_n})$, $\mathbf{V}(l)=(V_1,\dots, V_{N})$, let $W^n_i$ be the index of the vertex that the surplus edge corresponding to $V^n_i$ connects to. Similarly, let $W_i$ be the index of the vertex that the surplus edge corresponding to $V_i$ connects to. Then, 
\begin{align*}&\left(n^{-1/3}T^n_{l_n}, n^{-2/3}(l_n+1), \left(n^{-2/3}V^n_1,n^{-2/3}W^n_1\right) \dots, \left(n^{-2/3}V^n_{N_n}, n^{-2/3}W^n_{N_n}\right)\right)\\
&\todist\left(\cT_l, l, (V_1,W_1),\dots, (V_{N},W_{N})\right)\end{align*}
in the $2M+1$-pointed Gromov-Hausdorff topology. 
\end{proposition}
\begin{proof}
For $S$ a subset of the vertices of $T^n_{l_n}$, let $T^n_{l_n}(S)$ denote the subtree of $T^n_{l_n}$ spanned by $S$. By definition, for $m\leq N_n$, $W^n_m$ is the vertex corresponding to a uniform unpaired in-half-edge of the vertices in $T^n_{l_n}\left(\{l_n+1,V^n_1,\dots,V^n_{m}\}\right)$. By Lemma \ref{lemma.heightprocesswithlengths} and Slutsky's lemma,
$$\left(\frac{\hat{H}_n^\ell\left(\lfloor t n^{2/3}\rfloor \right)}{\hat{H}_n\left(\lfloor t n^{2/3}\rfloor \right)},t\geq 0\right)\overset{a.s.}{\to} \left(\frac{\sigma_{-+}+\nu_-}{2\mu},t\geq 0\right)$$
in $\D(\R_+,\R)$ as $n\to \infty$, which implies that the law of $W^n_m$ converges to the law of a uniform vertex in $T^n_{l_n}\left(\{l_n+1,V^n_1,\dots,V^n_{m}\}\right)$. 
Note that, by Theorem \ref{thm.convoutforest} and  Propositions \ref{prop.extractexcursions} and \ref{prop.convergencestartingpointscandidates}, we know that
$$\left(n^{-1/3}T^n_{l_n},n^{-2/3}l_n+1,n^{-2/3}V^n_1,\dots,n^{-2/3}V^n_{m}\right)\overset{a.s.}{\to}\left(\cT_l, l,V_1,\dots,V_m\right)$$
in the $m+1$-pointed Gromov-Hausdorff topology. Since the relation $$\left|T^n_{l_n}\left(\{l_n+1,V^n_1,\dots,V^n_{m}\}\right)\right|=\left|T^n_{l_n}\left(\{l_n+1,V^n_1,\dots,V^n_{m}, W^n_{m}\}\right)\right|$$ passes to the limit, with $|\cdot|$ denoting the length in the tree as encoded by $(\hat{H}_n(k),k\geq 1)$, the limit in distribution of $n^{-2/3}W^n_m$ is a uniform point on the subtree of $\cT_l$ spanned by $\left(l,V_1,\dots,V_m\right)$, which is equal to the law of $W_m$, which proves the statement.

\end{proof}
The proof of Propositions \ref{prop.convergencestartingpointscandidates} and \ref{prop.convergenceheadscandidates} implies the following corollary.
\begin{corollary}
By Skorokhod's representation theorem, we may work on a probability space where the convergence in Propositions \ref{prop.convergencestartingpointscandidates} and \ref{prop.convergenceheadscandidates} holds almost surely. Let $T^{n,\text{mk}}(l_n)$ be the subtree of $T^n_{l_n}$ spanned by $\{l_n+1,V^n_1,\dots,V^n_{N_n}\}$, and similarly, let $T^{\text{mk}}(l)$ be the subtree of $\cT_l$ spanned by $\{l,V_1,\dots,V_N\}$. Then, also 
\begin{align*}&\left(n^{-1/3}T^{n,\text{mk}}(l_n), n^{-2/3}(l_n+1), \left(n^{-2/3}V^n_1,n^{-2/3}W^n_1\right) \dots, \left(n^{-2/3}V^n_{N_n}, n^{-2/3}W^n_{N_n}\right)\right)\\
&\to \left(T^{\text{mk}}(l), l, (V_1,W_1),\dots, (V_{N},W_{N})\right)\end{align*}
almost surely in the $2M+1$-pointed Gromov-Hausdorff topology as $n\to \infty$. Also the total length in the trees converges, i.e.
$$n^{-1/3}\left|T^{n,\text{mk}}(l_n)\right|\to \left| T^{\text{mk}}(l)\right|$$
almost surely as $n\to\infty$.
\end{corollary}
We now identify the vertices that are part of a candidate as described in Subsection \ref{subsubsec.samplecandidates}. In $T^{n,\text{mk}}(l_n)$, set $V_i^n\sim W_i^n$ for each $1\leq i\leq N_n$, and set $M^n_{l_n}:=T^{n,\text{mk}}(l_n)/\sim$. Moreover, in $T^{\text{mk}}(l)$, set $V_i\sim W_i$ for each $1\leq i\leq N$, and set $\cM_l:=T^{\text{mk}}(l)/\sim$. View both as elements of $\vec{\cG}$ in the natural way. To be precise, in  $M^n_{l_n}$, let the vertex set consist of $l_n+1$, $W_i^n$ for $i\leq N_n$, and the branch points $V_i^n\wedge V_j^n$ for $i\neq j\leq N_n$. Similarly, in $\cM_l$, let the vertex set consist of $l$, $W_i$ for $i\leq N$, and the branch points $V_i\wedge V_j$ for $i\neq j\leq N$. Then, the following proposition follows.
\begin{proposition}
On the probability space where the convergence in Propositions \ref{prop.convergencestartingpointscandidates} and \ref{prop.convergenceheadscandidates} holds almost surely, 
$n^{-1/3}M^n_{l_n}\overset{a.s.}{\to} \cM_l$
in $\vec{\cG}$.
\end{proposition}
\begin{proof}
The proof is analogous to the proof of Proposition 5.6 in \cite{goldschmidtScalingLimitCritical2019}.
\end{proof}
\begin{corollary}\label{cor.sccsinonetreeconverge}
On the probability space where the convergence in Propositions \ref{prop.convergencestartingpointscandidates} and \ref{prop.convergenceheadscandidates} holds almost surely, the SCCs in $n^{-1/3}M^n_{l_n}$, listed in decreasing order of length, converge to the SCCs in $\cM_l$, listed in decreasing order of length, in $\vec{\cG}$ almost surely as $n\to \infty$.
\end{corollary}
\begin{proof}
This follows from Proposition 5.3 in \cite{goldschmidtScalingLimitCritical2019}. This proposition requires that the lengths of the SCCs in $\cM_{l}$ have different lengths almost surely, which is the content of Corollary \ref{cor.allengthsaredifferent}. 
\end{proof}

\begin{corollary}\label{cor.sccordereduptotimeT}
Let $T>0$, and let $(C^T_i(n),i\geq 1)$ be the SCCs in $(G_n(k),k\geq 1)$ with a candidate with label at most $\lfloor T n^{2/3}\rfloor$ ordered by length. Similarly, let $(\cC^T_i,i\geq 1)$ be the SCCs in $\vec{G}_n(\nu)$ with a candidate with label at most $T$ ordered by length. Then,
$$\left(n^{-1/3}C^T_i(n), i\geq 1\right) \todist (\cC^T_i,i\geq 1)$$
in the $\vec{\cG}$ product topology as $n\to \infty$. 
\end{corollary}
\begin{proof}
This follows from Proposition \ref{prop.extractexcursions}, Corollary \ref{cor.sccsinonetreeconverge}, and the fact that all SCCs in the limit object have a different length by Corollary \ref{cor.allengthsaredifferent}. 
\end{proof}

Finally, we claim that we can choose $T$ large enough such that all SCCs with large size are explored before time $T$. This is the content of the following lemma. The proof is in the same spirit as Lemma 9 in \cite{aldous_1991} by Aldous. 
\begin{lemma}\label{lemma.largesccfoundfirst}
For $\delta>0$ and $I$ an interval, let $SCC(n,I,\delta)$ denote the number of SCCs whose vertices have at total of at least $\delta n^{1/3}$ in-edges (including those which are not part of the SCC) and whose time of first discovery is in $n^{2/3}I$. Then,
$$\lim_{s\to \infty}\limsup_{n} \P\left(SCC(n,(s,\infty),\delta)\geq 1\right)=0\text{ for all }\delta>0.$$
\end{lemma}
\begin{proof}
Fix $\delta>0$. Suppose there is an SCC $C$ with $vn^{1/3}$ total in-edges. Conditionally on this fact, the in-edges that are paired up to the first in-edge of $C$ is paired, are uniform picks (without replacement) from the total set of in-edges. Denote the time of discovery of the first in-edge of $C$ times $n^{-2/3}$ by $\Xi_n$. Then, $\Xi_n\todist\operatorname{Exp}(v)$. Fix $\epsilon>0$. We have that, by the memoryless property at time $s$,
$$\P\left(SCC\left(n,(s,2s),\delta\right)=0|SCC\left(n,(s,\infty),\delta\right)\geq 1\right)$$
is asymptotically bounded from above by 
$\exp(-s\delta)$ by the memoryless property at time $s$, so that we can find an $s>0$ such that for all $n$ large enough,
$$\P\left(SCC\left(n,(s,\infty),\delta\right)\geq 1 \text{ and }SCC\left(n,(s,2s),\delta\right)=0\right)<\epsilon.$$
We claim that, by possibly increasing $s$ and $n$, we also get that 
$$\P\left(SCC\left(n,(s,2s),\delta\right)=0\right)>1-\epsilon,$$
which proves the statement.
Firstly, we observe that the ratio of the length of an $SCC$ and its total in-degree are asymptotically equal to $\frac{\sigma_{-+}+\nu_-}{2\mu}$ by the proof of Proposition \ref{prop.convergenceheadscandidates}. Then, note that it is clear from the description of the limit process that, for $s$ large enough, with probability at most $\epsilon/2$, an SCC with total length at least $\frac{\mu}{\sigma_{-+}+\nu_-}\delta$ is discovered after time $s$. By the convergence of the exploration process on compact time intervals, by choosing $n$ large enough, we can then ensure that 
$$\P\left(SCC\left(n,(s,2s),\delta\right)=0\right)>1-\epsilon.$$
We conclude that 
$$\P(SCC\left(n,(s,\infty),\delta\right)\geq 1)\leq 2\epsilon.$$
\end{proof}
Note that the size of an SCC is bounded from below by the total number of in-edges of vertices in the SCC. Then, Theorem \ref{thm.main} follows from Corollary \ref{cor.sccordereduptotimeT} and Lemma \ref{lemma.largesccfoundfirst}. 

% As discussed in Subsection \ref{subsec.categoriessurplusedged}, ancestral surplus edges are important building blocks of non-trivial SCCs, but also descendental and additional surplus edges can be included in an SCC. However, this can only occur if there is also an ancestral surplus edge present in the same SCCs. Corollary \ref{cor.convergencesequenceofcomponents} implies that the encoding processes of these components converge as a sequence ordered by component size (note that we include the purple vertices in the size). We will now define the process of finding the important additional and descendental surplus edges on such a component, given its encoding processes.\\
% We will identify purple vertices are possibly part of an SCC. We call these surplus edges \emph{candidates} and we define them in such a way that the important non-ancestral surplus edges are contained in the set of candidates. We will use Lemma \ref{lemma.whatispartofscc}.\ref{item.factsonsccs4} to define our candidates. The lemma states that non-ancestral surplus edges are only important if their head is on the path from the root of the out-component to the tail of an important surplus edge. This is not a sufficient condition, but by only declaring a surplus edge a candidate if it is on the path from the root of the out-component to the tail of another candidate, we will with high probability have a finite number of candidates per component, after which we can easily distinguish between candidates and important surplus edges.\\
% \subsubsection{Identifying the candidates}

% The procedure for finding candidates is as follows. Suppose we are given a finite tree $\cT$ with root $\rho$ and with $|\cT|$ vertices in total, in which the vertices are assigned indices in depth-first order, and in which some of the leaves are coloured purple. Moreover, suppose we have a set of marks $A=\left\{(x_i,y_i):1\leq i \leq m\right\}\subset \left(\{1,\dots,\cT\}\times \N\right)^m$ such that for each $i$ vertex $x_i$ is purple. These marks correspond to the ancestral surplus edges in the tree. We also set $A_k=\left\{(x_i,y_i):x_i\leq k, 1\leq i \leq m\right\}$. Let be $C_k$ the candidates found before time $k$. Then, for $B=\{(a_1,b_1),\dots,(a_k,b_k)\}$ a subset of $[0,|\cT|]\times \N$, let $\cT(B)$ be the subtree of $\cT$ spanned by $\{\rho,a_1,\dots,a_k\}$. We define the active graph at time $k$ to be $\cG_k=\cT(A_k\cup C_k)\backslash \cT(\{k\})$. Note that by the argument above, a non-ancestral surplus edge visited at time $k$ should be declared a candidate if and only if its head is in $\cG_k$. \\
% We need extra information to determine the probability that the head of a non-ancestral surplus edge visited at time $k$ has its head in $\cG_k$. Like in Subsection \ref{subsec.ancestral}, we equip $\cT$ with edge lengths to encode the number of in-edges of vertices in $\cT$. Denote the resulting tree by $\cT^\ell$, and for a subgraph $\cG$ of $\cT$, let $\cG^\ell$ be equal to $\cG$ seen as a subtree of $\cT^\ell$ and let $\ell(\cG)$ the total length of $\cG^\ell$. In our model, $\cT$ will play the rôle of one tree in a forest, and we need information on trees explored before $\cT$, because surplus edges can also connect to those trees. We encode this with a process $(\mathsf{s}^-_{|\cT|}(k),1\leq k \leq |\cT|)$, that encodes the total number of seen available in-edges, such that $\mathsf{s}^-_{|\cT|}(0)$ represents the number of available in-edges in components explored before $\cT$ when we start the exploration of $\cT$. Then, the procedure is defined as follows. Perform the following iterative procedure for $k\in(1,\dots, |\cT|)$. Set $C_1=\emptyset$.
% \begin{enumerate}
%     \item If $k$ is not purple, or $(k,y)\in A_k$ for some $y$, set $C_k=C_{k+1}$.
%     \item Otherwise, with probability 
%     $$\frac{\ell(\cG_k)}{\mathsf{s}^-_{|\cT|}(k)}$$
%     declare $k$ a candidate. Pick a uniform point in $\cG^\ell$ according to the length measure, and let $y$ be the first vertex on the path to $\rho$ from this point. Set $C_{k+1}=C_k\cup\{(k,y)\}$
% \end{enumerate}
% Set $C(\cT)=C_{|\cT|}$. 



% \subsubsection{Convergence of the positions of the candidates}
% \begin{proposition}

% \end{proposition}





% For each $n\in \N$, fix $\sigma_n \in \R$, and let $\left(\mathsf{h}^n,\mathsf{h}^{\ell,n}, \mathsf{s}^{-,n}, \mathsf{p}^n, \mathsf{n}^n, \mathsf{u}^n\right)$ be a function from $[0,\sigma_n]$ to $\R_+^4\times \N \times (\R_+\cup \{\omega\})$. Similarly, let $\sigma \in \R$, and let $\left(\mathsf{h},\mathsf{h}^{\ell}, \mathsf{s}^{-}, \mathsf{p}, \mathsf{n}, \mathsf{u}\right)$ be a function from $[0,\sigma]$ to $\R_+^4\times \N \times (\R_+\cup \{\omega\})$. Assume that $\mathsf{n}^n$, $\mathsf{p}^n$, $\mathsf{n}$ and $\mathsf{p}$ are increasing functions that satisfy $\mathsf{n}^n(0)=\mathsf{p}^n(0)=\mathsf{n}(0)=\mathsf{p}(0)=0$. Also assume that $\mathsf{n}^n$ and $\mathsf{n}$ have increments of size $1$, and that $\mathsf{u}^n(t)\neq \omega$ if and only if $\mathsf{n}^n(t)-\mathsf{n}^n(t-)=1$, in which case $$\mathsf{u}^n(t)<\mathsf{h}^{n}(t)$, and similarly, that $\mathsf{u}(t)\neq \omega$ if and only if $\mathsf{n}(t)-\mathsf{n}(t-)=1$, in which case $$\mathsf{u}(t)<\mathsf{h}(t)$. Suppose that

% \subsubsection{Important surplus edges on a single tree}