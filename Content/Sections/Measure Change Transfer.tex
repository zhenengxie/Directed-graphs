\chapter{The measure change}
\label{chap:measure-change}

We first recall some notation. Recall $\vD_1, \ldots, \vD_n$ are i.i.d.\ copies of $\vD = (D^-, D^+)$. We assume
\begin{enumerate}
    \item $\E[D^-] = \E[D^+]$, call this quantity $\mu$.
    \item $D^- - D^+$ is \emph{strongly aperiodic}. This means $D^- - D^+$ has main lattice $\Z$.
    \item $\E[\abs{D^-}^3], \E[\abs{D^+}^3] < \infty$.
    \item $\Pr(D^- = 0) = 0$.
\end{enumerate}
Further $\vZ_1, \vZ_2, \ldots$ is an i.i.d.\ sequence such that $\vZ_i = (Z_i^-, Z_i^+)$ has law
\begin{equation*}
    \Pr(\vZ_i^- = k^-, \vZ_i^+ - k^+) \defeq \frac{k^-}{\mu} \Pr(D^- = k^-, D^+ = k^+).
\end{equation*}
We use the notation
\begin{equation*}
    \nu_{\pm} \defeq \E[Z_i^{\pm}].
\end{equation*}
Next we restate the theorem we are trying to prove:
\measurechange*
Note the criticality condition is unnecessary here.

\section{Exact form of measure change}

We begin by finding the explicit form of the measure change. We will use the notation
\begin{equation*}
    \Xi_{n-m}^{\pm} = \textstyle \sum_{i=m+1}^n D_i^{\pm}
    \quad \text{and} \quad
    \Delta_{n-m} = \Xi_{n-m}^- - \Xi_{n-m}^+.
\end{equation*}
The following lemma achieves this when there is no conditioning on equal total in- and out-degrees.
\begin{lemma}
    \label{lem:unconditioned-mc}
    For any test function $u: (\N \times \N)^m \to \R$, 
    \begin{equation*}
        \E[u(\vect{\biasD}_{n, 1}, \ldots, \vect{\biasD}_{n, m})] =
        \E[u(\vect{Z}_1, \ldots, \vect{Z}_m) \unconmc^n_m(\vect{Z}_1, \ldots, \vect{Z}_m)]
    \end{equation*}
    where
    \begin{equation*}
        \unconmc^n_m(\vect{k}_1, \ldots, \vect{k}_m) \defeq
        \E \left[ \prod_{i=1}^m \frac{(n - i + 1)\mu}{\sum_{j=i}^n k_j^- + \Xi^-_{n - m}}\right].
    \end{equation*}
\end{lemma}
The proof of this for the undirected case is covered in \cite[Proposition 4.2]{conchon--kerjanStableGraphMetric2020} and the same proof translates straightforwardly to the directed case. Using this lemma we can obtain the measure change after the conditioning.
\begin{lemma}
    \label{lem:measure-change-exact-conditioned}
    For any test function $u: (\N \times \N)^m \to \R$, 
    \begin{equation}
        \E\left[u(\vect{\biasD}_{n, 1}, \ldots, \vect{\biasD}_{n, m}) \mid \Delta_n = 0 \right] =
        \E[u(\vect{Z}_1, \ldots, \vect{Z}_m) \conmc^n_m(\vect{Z}_1, \ldots, \vect{Z}_m)]
        \label{eq:conditioned-measure-change-condition}
    \end{equation}
    where
    \begin{equation*}
        \conmc^n_m(\vect{k}_1, \ldots, \vect{k}_m) = \frac{
        \E \left[ \prod_{i=1}^m \frac{(n - i + 1)\mu}{\sum_{j=i}^n k_j^- + \Xi^-_{n - m}}
        \indi \left\{ \Delta_{n-m} = \sum_{i=1}^m (k^+_i - k^-_i) \right\}\right]
        }{\Pr(\Delta_n = 0)}.
    \end{equation*}
\end{lemma}

\begin{proof}
    By \cref{lem:unconditioned-mc}, since
    \begin{equation*}
        \unconmc^n_n(\vect{k}_1, \ldots, \vect{k}_m) =
        \prod_{i=1}^n \frac{(n - i + 1) \mu}{\sum_{j=i}^n k_j^-}
    \end{equation*}
    we have
    \begin{align*}
        &\E\left[u(\vect{\biasD}_{n, 1}, \ldots, \vect{\biasD}_{n, m}) \indi\left\{ \sum_{i=1}^n (\biasD^-_{n, i} - \biasD^+_{n, i}) = 0 \right\} \right] \\
        = &\E\left[u(\vect{Z}_1, \ldots, \vect{Z}_m) \indi\left\{ \sum_{i=1}^n (Z_i^- - Z_i^+) = 0 \right\} \prod_{i=1}^n \frac{(n - i + 1) \mu}{\sum_{j=i}^n Z_j^-} \right] \\
        = &\E\Bigg[u(\vect{Z}_1, \ldots, \vect{Z}_m) \E \Bigg[ \indi\left\{ \sum_{i=m+1}^n (Z_i^- - Z_i^+) = \sum_{i=1}^m (Z^+_i - Z^-_i) \right\} \nonumber \\
       & \hspace{5.5em} \times \prod_{i=1}^m \frac{(n - i + 1)\mu}{\sum_{j=i}^m Z_j^- + \sum_{j = m+1}^n Z_j^-} \prod_{i=m+1}^n \frac{(n - i + 1) \mu}{\sum_{j=i}^n Z_j^-} \mid \vect{Z}_1, \ldots, \vect{Z}_m \Bigg] \Bigg].
    \end{align*}
    Thus \cref{eq:conditioned-measure-change-condition} holds with
    \begin{align*}
        \conmc^n_m(\vect{k}_1, \ldots, \vect{k}_m)
        &= \frac{1}{\Pr(\Delta_n = 0)}\E\Bigg[ \indi\left\{ \sum_{i=m+1}^n (Z_i^- - Z_i^+) = \sum_{i=1}^m (k^+_i - k^-_i)\right\} \nonumber \\
        &\hspace{6em} \times \prod_{i=1}^m \frac{(n - i + 1)\mu}{\sum_{j=i}^m k_j^- + \sum_{j = m+1}^n Z_j^-} \prod_{i=m+1}^n \frac{(n - i + 1) \mu}{\sum_{j=i}^n Z_j^-} \Bigg].
    \end{align*}
    Since the $Z_i$ are i.i.d.\ we can shift indices to get
    \begin{align*}
        \conmc^n_m(\vect{k}_1, \ldots, \vect{k}_m)
        &= \frac{1}{\Pr(\Delta_n = 0)}\E\Bigg[ \indi\left\{ \sum_{i=1}^{n-m} (Z_i^- - Z_i^+) = \sum_{i=1}^m (k^+_i - k^-_i)\right\} \nonumber \\
        &\hspace{5.5em} \times \prod_{i=1}^m \frac{(n - i + 1)\mu}{\sum_{j=i}^m k_j^- + \sum_{j = 1}^{n -m} Z_j^-} \prod_{i=1}^{n-m} \frac{(n - m - i + 1) \mu}{\sum_{j=i}^{n - m} Z_j^-} \Bigg].
    \end{align*}
    Note that
    \begin{equation*}
        \prod_{i=1}^{n-m} \frac{(n - m - i + 1) \mu}{\sum_{j=i}^{n - m} Z_j^-} = \unconmc^{n-m}_{n-m}(\vect{Z}_1, \ldots, \vect{Z}_{n-m})
    \end{equation*}
    therefore by \cref{lem:unconditioned-mc}
    \begin{align*}
        \conmc^n_m(\vect{k}_1, \ldots, \vect{k}_m)
        &= \frac{1}{\Pr(\Delta_n = 0)}\E\Bigg[ \indi\left\{ \sum_{i=1}^{n-m} (\biasD_{n-m, i}^- - \biasD_{n-m, i}^+) = \sum_{i=1}^m (k^+_i - k^-_i)\right\} \nonumber \\
        &\hspace{9em} \times \prod_{i=1}^m \frac{(n - i + 1)\mu}{\sum_{j=i}^m k_j^- + \sum_{j = 1}^{n -m} \biasD_{n-m, j}^-} \Bigg].
    \end{align*}
    Finally since
    \begin{equation*}
        \sum_{i=1}^{n-m} \biasD^{\pm}_{n-m, i} = \sum_{i=1}^{n-m} D^{\pm}_i \eqdist \sum_{i=m+1}^n D^{\pm}_i = \Xi_{n-m}^{\pm}
    \end{equation*}
    we obtain the desired result of
    \begin{equation*}
        \conmc^n_m(\vect{k}_1, \ldots, \vect{k}_m) = \frac{
        \E \left[ \prod_{i=1}^m \frac{(n - i + 1)\mu}{\sum_{j=i}^n k_j^- + \Xi^-_{n - m}}
        \indi \left\{ \Delta_{n-m} = \sum_{i=1}^m (k^+_i - k^-_i) \right\}\right]
        }{\Pr(\Delta_n = 0)}.
    \end{equation*}
\end{proof}

\section{Asymptotic bound of the measure change from below}

In \cref{thm:measure-change} we see the correct scaling to in order to get a sensible scaling limit is $m = \Theta(n^{2/3})$. This will be used to prove the out-components scale as $n^{2/3}$. The following result, which is an analogue of \cite[Lemma 6.7]{conchon--kerjanStableGraphMetric2020}, describes the asymptotic behaviour of the measure change when $m = \Theta(n^{2/3})$.
\begin{lemma}
    \label{lem:measure-change-approx}
    Define
    \begin{equation*}
        s^{\pm}(i) \defeq \textstyle{\sum_{j=1}^i (k_i^{\pm} - \nu_{\pm})}.
    \end{equation*}
    Suppose we are given $\epsilon \in (0, 1/6)$ and that $\vect{k}_1, \ldots, \vect{k}_m$ are such that
    \begin{equation}
        \label{eq:s-condition}
        \max_{i=1, \ldots, m} \abs{s^{\pm}(i)} \leq m^{\frac{1}{2} + \epsilon}.
    \end{equation}
    Then in the regime $m = \Theta(n^{2/3})$
    \begin{equation*}
        \conmc^n_m(\vect{k}_1, \ldots, \vect{k}_m)
        \geq \exp\left( \frac{1}{\mu n} \sum_{i=0}^m (s^-(i) - s^-(m)) - \frac{\biasvar}{6 \mu^2} \frac{m^3}{n^2} \right) + \littleo(1)
    \end{equation*}
    where the $\littleo(1)$ term is independent of $\vect{k}_1, \ldots, \vect{k}_m$ satisfying our assumptions.
\end{lemma}
The $1/6$ upper bound on $\epsilon$ is used to show certain terms that arise in the proof will decay to 0, but the exact value of $1/6$ is unimportant. Rather $\epsilon$ should be thought of as a positive constant we can make arbitrarily small to make the proof work. We also explain the condition in \cref{eq:s-condition}. In \cref{thm:measure-change} we evaluate $\phi^n_m$ as 
\begin{equation*}
    \phi^n_m(\vZ_1, \ldots, \vZ_m).
\end{equation*}
Thus the condition in \cref{eq:s-condition} corresponds to the event
\begin{equation*}
    \max_{i=1, \ldots, m } \abs*{
        \sum_{j=1}^i (Z^{\pm}_j - \nu_{\pm}) 
    } \leq m^{1/2 + \epsilon}.
\end{equation*}
This is saying that the centered random walks corresponding to $Z^+_i$ and $Z^-_i$ do not deviate by more than $m^{1/2 + \epsilon}$ in the first $m$ steps. This will occur with high probability and thus \cref{eq:s-condition} is not a restrictive condition to take.

The fact that we only prove a lower bound may seem strange at first. The idea is that since we are dealing with a measure change, as long as the lower bound will have expectation 1 in the limit, this shows that we have not lost a significant amount of mass. The following lemma taken from \cite[Lemma 4.8]{conchon--kerjanStableGraphMetric2020} makes this formal.
\begin{lemma}
    \label{lem:lower-bound-suff}
    Let $(X_n)_{n \geq 1}$ and $(Y_n)_{n \geq 1}$ be two sequences of non-negative random variables such that $X_n \geq Y_n$ and $\E[X_n] = 1$ for all $n$. Suppose there exists another non-negative random variable $X$ such that $Y_n \todist X$ and $\E[X] = 1$. Then $X_n \todist X$ and $(X_n)_{n \geq 1}$ is uniformly integrable.
\end{lemma}

\subsection{Exponential tilting}

Note that
\begin{equation*}
    \E[Z^- - Z^+] = \frac{1}{\mu} \E[D^-D^+ - (D^-)^2]
\end{equation*}
and thus is, in general, non-zero even if $\E[D^- - D^+] = 0$. The deviation of the $Z^{\pm}_i$ around their mean is controlled by the assumption in \cref{eq:s-condition}. If $\vk_1, \ldots, \vk_n$ satisfy \cref{eq:s-condition} and $m = \Theta(n^{2/3})$ then
\begin{equation*}
    \sum_{i=1}^m (k_i^- - k_i^+) = s^-(m) - s^+(m) + (\nu_+ - \nu_-) m = \Theta(n^{2/3}).
\end{equation*}

In contrast $\Delta_{n-m}$ is centered, so
\begin{equation*}
    \left\{ \Delta_{n-m} = \textstyle \sum_{i=1}^m (k_i^+ - k_i^-) \right\}  
\end{equation*}
is looking at the event that $\Delta_{n-m}$ takes a value at distance $n^{2/3}$ away from its mean. As remarked in \cref{chap:llt}, the local limit theorem provides no information at distances $\omega(n^{1/2})$ from the mean since the error term will dominate.

To shift the mean of $\Delta_{n-m}$ we will introduce a sequence of exponentially tilted measures. The next result defines this tilt and then gives asymptotic expansions for cumulant generating function of $D^-$, the mean of $D^-$ and the mean of $D^+$ under this tilting. 
\begin{lemma}
    \label{lem:asym-expansions}
    Define an measure $\Pr_{\theta}$, for $\theta \geq 0$, by its Radon--Nikodym derivative
    \begin{equation*}
        \diff{\Pr_{\theta}}{\Pr} \defeq \exp\left( - \theta D^- - \alpha(\theta) \right)
        \quad \text{where} \quad
        \alpha(\theta) \defeq \log \E \left[ e^{-\theta D^-} \right].
    \end{equation*}
    Then as $\theta \downarrow 0$ we have
    \begin{align*}
        \alpha(\theta) &= -\mu \theta + \tfrac{1}{2}\var(D^-) \theta^2 - \tfrac{1}{6} \E \left[ (D^- - \mu)^3 \right] \theta^3 + \littleo(\theta^3), \\
        \E_{\theta}[D^-] &= \mu - \var(D^-) \theta + \bigo(\theta^3), \\
        \text{and} \quad \E_{\theta}[D^+] &= \mu - \cov(D^-, D^+) \theta + \bigo(\theta^3).
    \end{align*}
\end{lemma}
\begin{proof}
    Since $\E\left[\abs{D^-}^3\right] < \infty$ and $D^-$ is non-negative, by the dominated convergence theorem
    \begin{equation}
        \E \left[ (D^-)^3 \exp(-\theta D^-) \right] = \E \left[ (D^-)^3 \right] + \littleo(1)
        \label{eq:mgf-start}
    \end{equation}
    as $\theta \downarrow 0$. Integrating \cref{eq:mgf-start} with respect to $\theta$ and applying Fubini's theorem gives
    \begin{align}
        \int_0^{\theta} \E \left[ (D^-)^3 e^{-\theta' D^-}\right] \dif \theta'
        &= \int_0^{\theta} (\E \left[ (D^-)^3 \right] + \littleo(1) ) \dif \theta \nonumber \\
        \iff \E \left[ \int_0^{\theta} (D^-)^3 e^{-\theta' D^-} \dif \theta' \right]
        &= \E \left[ (D^-)^3 \right] \theta + \littleo(\theta) \nonumber \\
        \iff \E \left[ (D^-)^2 \right] - \E \left[ (D^-)^2 e^{-\theta D^-} \right]
        &= \E \left[ (D^-)^3 \right] \theta + \littleo(\theta)  \nonumber \\
        \iff \E \left[ (D^-)^2 e^{-\theta D^-} \right]
        &= \E \left[ (D^-)^2 \right] - \E \left[ (D^-)^3 \right] \theta + \littleo(\theta). \nonumber
    \end{align}
    Repeating this method yields
    \begin{align}
        \E \left[ D^- e^{-\theta D^-} \right]
        &= \mu - \E \left[ (D^-)^2 \right] \theta + \tfrac{1}{2} \E \left[ (D^-)^3 \right] \theta^2 + \littleo(\theta^2), \label{eq:asymin} \\
        \text{and} \quad \E \left[ e^{-\theta D^-} \right]
        &= 1 - \mu \theta + \tfrac{1}{2} \E \left[ (D^-)^2 \right] \theta^2 - \tfrac{1}{6} \E \left[ (D^-)^3 \right] \theta^3 + \littleo(\theta^3).
        \label{eq:asym00}
    \end{align}
    Similarly integrating the equation
    \begin{equation*}
        \E \left[ (D^-)^2 D^+ \exp(-\theta D^-) \right] = \E \left[ (D^-)^2 D^+ \right] + \littleo(1)
    \end{equation*}
    twice gives
    \begin{equation}
        \E \left[ D^+ e^{-\theta D^-} \right]
        = \mu \theta - \E \left[ D^- D^+ \right] \theta + \tfrac{1}{2} \E \left[ (D^-)^2 D^+ \right] \theta^2 + \littleo(\theta^2).
        \label{eq:asymout}
    \end{equation}
    \cref{eq:asym00} gives the expansion of the normalising constant of the measure change. Combining this with \cref{eq:asymin} and \cref{eq:asymout} yields the expansions for $\E_{\theta}[D^-]$ and $\E_{\theta}[D^+]$ respectively. Taking the logarithm of \cref{eq:asym00} gives the expansion of the cumulant generating function $\alpha(\theta)$.
\end{proof}

To achieve the centering of $\Delta_{n-m}$ we desire, let us define a sequence of tilted measures $\Pr_n$ defined by their Radon--Nikodym derivative
\begin{equation}
    \diff{\Pr_n}{\Pr} \defeq \exp \left( - \theta_n \Xi^-_{n-m} - (n - m) \alpha(\theta_n) \right)
\end{equation}
where $\theta_n \defeq \frac{m}{\mu n}$. This factorises and so $\vD_1, \ldots, \vD_n$ remain i.i.d.\ under this tilting and each $\vD_i$ has the law of $\vD$ under $\Pr_{\theta_n}$. Applying \cref{lem:asym-expansions} we can compute that
\begin{align*}
    \E_n[\Delta_{n-m}] 
    &= m(\nu_+ - \nu_-) + \bigo(n^{1/3}).
\end{align*}
Hence
\begin{align*}
    \textstyle \sum_{i=1}^m (k_i^+ - k_i^-) - \E_n[\Delta_{n-m}] 
    &= s^-(m) - s^+(m) + \Big[ m(\nu_+ - \nu_-) - \E_n[\Delta_{n-m}] \Big] \\
    &= \bigo(n^{1/3 + \epsilon})
\end{align*}
which is within the $\bigo(n^{1/2})$ range from the mean where the local limit theorem yields useful information. Thus $\theta_n = \frac{m}{\mu n}$ is the correct tilting to take. 

\subsection{Local limit results}

We wish to determine the local limit behaviour of $\Delta_n$ under the untilted measure and $\Delta_{n-m}$ under the tilted measures. In the latter case, we wish to show the behaviour remains the same even when conditioning on $\Xi_{n-m}^-$ having fluctuations of order $\bigo(n^{1/2 + \epsilon})$ about its tilted mean. The way we show this is to first show a bivariate local limit theorem.

$(D^- - D^+, D^-)$ is $\Z^2$ valued and thus certainly lattice valued. Further in the degenerate case where $D^- = D^+$ almost surely, the total in-degree and total out-degree will be equal almost surely and thus the conditioned and non-conditioned case are the same. So the proof of \cref{thm:measure-change} will be the same as for \cite[Proposition 4.3]{conchon--kerjanStableGraphMetric2020}. Hence WLOG assume $(D^- - D^+, D^-)$ is non-degenerate. Let $\lattice$ be the main lattice of $(D^- - D^+, D^-)$. Then by \cref{lem:zd-triangular} there exist $p, q, r \in \Z_{\geq 0}$ such that $\lattice$ is generated by the columns of
\begin{equation*}
    \begin{pmatrix}
        p & 0 \\
        r & q
    \end{pmatrix}
\end{equation*}
We have assumed that $D^- - D^+$ has main lattice $\Z$ and therefore $p = 1$. Conditional on $D^- - D^+$ taking some value, $D^-$ will have main lattice $q \Z$. Let $\sigma^2$ be the variance of $D^- - D^+$ and $\Sigma$ be the covariance matrix of $(D^- - D^+, D^-)$.

The $\Pr(\Delta_n = 0)$ divisor in the definition of $\phi_m^n$ can be handled by the standard local limit theorem.
\begin{lemma}
    \label{lem:normal-llt}
    As $n \to \infty$
    \begin{equation*}
        \Pr(\Delta_n = 0) = \frac{1}{\sqrt{2 \pi \sigma^2 n}} \left( 1 + \littleo(1) \right).
    \end{equation*}
\end{lemma}
\begin{proof}
    Let $X = D^- - D^+$. This is assumed to have finite variance $\sigma$ and main lattice $\Z$. Moreover $X$ is centered and $0$ is in the main lattice $\Z$. Thus the result follows by \cref{thm:normal-multi-llt}.
\end{proof}

Next we show the local limit theorem holds for $(\Delta_{n-m}, \Xi^-_{n-m})$ under the tilting.
\begin{lemma}
    \label{lem:bivar-llt}
    Let $\vc_n \in \Z^2$ be such that $\vc_n + \Lambda$ contains the support of $(\Delta_{n-m}, \Xi^-_{n-m})$. Then uniformly for $(x, y) \in \vc_n + \Lambda$
    \begin{align*}
        &\Pr_n\left(
            \Delta_{n-m} = \E \big[ \Delta_{n-m} \big] + x, \ 
            \Xi^-_{n-m} = \E \big[ \Xi^-_{n-m} \big] + y
        \right)  \nonumber \\
        &\hspace{7em} = \frac{q}{2\pi \det(\Sigma)^{1/2} \: n} \exp\left( 
            \frac{-1}{2n}
            \begin{pmatrix}
                x & y
            \end{pmatrix}
            \Sigma
            \begin{pmatrix}
                x \\ y
            \end{pmatrix}
         \right)
         + \littleo( n^{-1} )
    \end{align*}
    as $n \to \infty$.
\end{lemma}
\begin{proof}
    Let $\vX_{n1}, \ldots, \vX_{nn}$ have the same distribution as 
    \begin{equation*}
        \vectwo{D^-_1 - D^+_1}{D^-_1}, \ldots, \vectwo{D^-_n - D^+_n}{D^-_n}
    \end{equation*}
    under the tilted measure $\Pr_n$. Then since $\theta_n = \littleo(1)$, it is simple to show that $\vX_{n1}$ tends weakly in law to $\vX \defeq (D^- - D^+, D^-)$ under the non-tilted measure. The measure change does not change the support of the random variables and thus all $\vX_{n1}$ and $\vX$ have the same main lattice $\lattice$. Finally we check the uniform integrability condition. Since $\theta_n = \littleo(1)$, $M = - \inf_n \alpha(\theta_n) < \infty$. Then
    \begin{align*}
        \sup_n \E[\norm{\vX_{n1}}^2 \indi \{ \norm{\vX_{n1}}^2 \geq L \} ]
        &= \sup_n \E[e^{-\theta_n D^- - \alpha(\theta_n)} \norm{\vX}^2 \indi\{\norm{\vX}^2 \geq L \} ] \\
        &\leq e^M \E[\norm{\vX}^2 \indi\{\norm{\vX}^2 \geq L \} ] \\
        & \to 0
    \end{align*}
    as $L \to \infty$ since $\vX$ is assumed to have finite second moment. Thus the desired result follows from \cref{thm:multi-triangular-llt}. There is a slight change in that we have $n - m$ terms in the summations for $\Delta_{n-m}$ and $\Xi^-_{n-m}$ rather than $n$ terms, but since $m = \Theta(n^{2/3})$ this will not matter asymptotically.
\end{proof}

Now we show the $\Pr(\Delta_{n-m} = \sum_{i=1}^m (k_i^+ - k_i^-))$ will have the same asymptotic behaviour as $\Pr(\Delta_n = 0)$ even when we condition on $\Xi_{n-m}^-$ not `varying too much' about its tilted mean. We only prove a lower bound, but this is sufficient for proving \cref{lem:measure-change-approx}.
\begin{lemma}
    \label{lem:mod-dev-local}
    Under the assumptions of \cref{lem:measure-change-approx},
    \begin{equation*}
        \mathbb P_n \left(
            \Delta_{n-m} = \sum_{i=1}^m (k_i^+ - k_i^-), \ 
            \abs{\Xi^-_{n-m} - \E_n[\Xi^-_{n-m}]} \leq n^{\frac{1}{2} + \epsilon}
        \right)
        \geq \frac{1}{\sqrt{2 \pi \sigma^2 n}} (1 + \littleo(1))
    \end{equation*}
\end{lemma}
\begin{proof}
    For convenience let
    \begin{equation*}
        P_n \defeq \mathbb P_n \left(
            \Delta_{n-m} = \sum_{i=1}^m (k_i^+ - k_i^-), \ 
            \abs{\Xi^-_{n-m} - \E_n[\Xi^-_{n-m}]} \leq n^{\frac{1}{2} + \epsilon}
        \right).
    \end{equation*}
    Define
    \begin{equation*}
        a_n \defeq \sum_{i=1}^m (k_i^+ - k_i^-) - \E_n[\Delta_{n-m}].
    \end{equation*}
    Also let
    \begin{equation*}
        L_n \defeq \Big\{
            y : \bigg( \textstyle\sum_{i=1}^m (k_i^+ - k_i^-), y \bigg) \in \vc_n + \lattice
            \Big\}.
    \end{equation*}
    $L_n$ has a simpler representation. Fix any $y_0 \in L_n$. Then
    \begin{equation*}
        \text{$\lattice$ is generated by the columns of} \begin{pmatrix}
            1 & 0 \\
            r & q
        \end{pmatrix}
        \implies L_n = y_0 + q \Z.
    \end{equation*}
    Fix arbitrary $M > 0$. Then
    \begin{align*}
        P_n &= \sum_{\substack{y \in L_n \\ \abs{y} \leq n^{1/2 + \epsilon}}} \mathbb P_n \left( \Delta_{n-m} = \E_n[\Delta_{n-m}] + a_n, \ \Xi^-_{n-m} = \E_n[\Xi^-_{n-m}] + y \right) \\
        &\geq \sum_{\substack{y \in L_n \\ \abs{y} \leq M n^{1/2}}} \mathbb P_n \left( \Delta_{n-m} = \E_n[\Delta_{n-m}] + a_n, \ \Xi^-_{n-m} = \E_n[\Xi^-_{n-m}] + y \right)
    \end{align*}
    for all $n$ sufficiently large. By \cref{lem:bivar-llt}, using that the error is uniform, we have that
    \begin{equation*}
        P_n \geq \sum_{\substack{y \in L_n \\ \abs{y} \leq M n^{1/2}}} 
         \frac{q}{2\pi \det(\Sigma)^{1/2} \: n} \exp\left( 
            \frac{-1}{2n} \vectwo{a_n}{y} \cdot \Sigma^{-1} \vectwo{a_n}{y}
         \right)
         + \littleo( n^{-1/2} )
    \end{equation*}

    To factorise this we make a change of variables. There exists $c$ such that
    \begin{equation*}
        \cov(D^- - c(D^- - D^+), D^- - D^+) = 0.
    \end{equation*}
    Let $\tau^2$ be the variance of $D^- - c(D^- - D^+)$. Then
    \begin{align*}
         &\frac{q}{2\pi \det(\Sigma)^{1/2} \: n} \exp\left( 
            \frac{1}{2n} \vectwo{a_n}{y} \cdot \Sigma^{-1} \vectwo{a_n}{y}
         \right) \nonumber \\
         & \hspace{5em} =
         \frac{1}{\sqrt{2 \pi \sigma^2 n}} \exp \left( - \frac{1}{2 \sigma^2} \frac{a_n^2}{n} \right)
         \frac{q}{\sqrt{2 \pi \tau^2 n}} \exp \left( - \frac{1}{2 \tau^2} \frac{(y - ca_n)^2}{n} \right).
    \end{align*}
    We now examine the asymptotic behaviour of $a_n$. By \cref{lem:asym-expansions},
    \begin{align*}
        \E_n[\Delta_{n-m}]
        &= (n - m) \E_{\theta_n}[D^- - D^+] \\ 
        &= -(\nu_- - \nu_+)m + \bigo(n^{1/3}).
    \end{align*}
    Therefore 
    \begin{equation*}
        a_n = s_+(m) - s_-(m) + \bigo(n^{1/3}) = \bigo(n^{1/3 + \epsilon})
    \end{equation*}
    by the assumption in \cref{eq:s-condition}. Thus
    \begin{equation*}
        P_n \geq
        \frac{1}{\sqrt{2 \pi \sigma^2 n}}(1 + \littleo(1))
        \sum_{\substack{y \in L_n \\ \abs{y} \leq M n^{1/2}}}
        \frac{q}{\sqrt{2 \pi \tau^2 n}} \exp \left( - \frac{1}{2 \tau^2} \frac{(y - ca_n)^2}{n} \right)
        + \littleo(n^{-1/2})
    \end{equation*}
    Note that
    \begin{equation*}
        \sum_{\substack{y \in L_n \\ \abs{y} \leq M n^{1/2}}}
        \frac{q}{\sqrt{2 \pi \tau^2 n}} \exp \left( - \frac{1}{2 \tau^2} \frac{(y - ca_n)^2}{n} \right)
        = \sum_{\substack{y \in L_n \\ \abs{y} \leq M n^{1/2}}}
        \frac{q}{\sqrt{n}}\ g\left( \frac{y - ca_n}{\sqrt{n}} \right)
    \end{equation*}
    where
    \begin{equation*}
        g(z) = \frac{1}{\sqrt{2 \pi \tau^2}} \exp\left( \frac{-z^2}{2 \tau^2} \right).
    \end{equation*}
    Since $a_n = \bigo(n^{1/3 + \epsilon})$, for $n$ sufficiently large
    \begin{align}
        \sum_{\substack{y \in L_n \\ \abs{y} \leq M n^{1/2}}}
        \frac{q}{\sqrt{2 \pi \tau^2 n}} \exp \left( - \frac{1}{2 \tau^2} \frac{(y - ca_n)^2}{n} \right)
        &\geq \sum_{\substack{z \in L_n - ca_n\\ \abs{z} \leq \frac{1}{2} M n^{1/2}}} 
        \frac{q}{\sqrt{n}}\ g \left( \frac{z}{\sqrt{n}} \right) \\
        &= \sum_{\substack{z \in \tilde{L}_n \\ \abs{z} \leq \frac{1}{2} M }}
        \frac{q}{\sqrt{n}}\ g(z) \label{eq:riemann-sum}
    \end{align}
    where
    \begin{equation*}
        \tilde{L}_n = \frac{L_n - ca_n}{\sqrt{n}}.
    \end{equation*}
    Then $\tilde{L}_n \cap [-\frac{1}{2}M, \frac{1}{2}M]$ is a partition of $[-\frac{1}{2}M, \frac{1}{2}M]$ where adjacent points are distance $q/\sqrt{n}$ apart from each other. Thus \cref{eq:riemann-sum} is a Riemann sum approximation of an integral. Hence
    \begin{equation*}
        \sum_{\substack{y \in L_n \\ \abs{y} \leq M n^{1/2}}}
        \frac{q}{\sqrt{2 \pi \tau^2 n}} \exp \left( - \frac{1}{2 \tau^2} \frac{(y - ca_n)^2}{n} \right)
        \geq (1 + \littleo(1)) \int_{-\frac{1}{2} M}^{\frac{1}{2}M} g(z) \dif z.
    \end{equation*}
    Thus
    \begin{equation*}
        P_n \geq \frac{1}{\sqrt{2 \pi \sigma^2 n}} (1 + \littleo(1)) \int_{-\frac{1}{2} M}^{\frac{1}{2}M} g(z) \dif z.
    \end{equation*}
    This holds for all $M > 0$ and $\int_{-\infty}^{\infty} g(z) \dif z = 1$ therefore 
    \begin{equation*}
        P_n \geq \frac{1}{\sqrt{2 \pi \sigma^2 n}} \left( 1 + \littleo(1) \right)
    \end{equation*}
    as required.
\end{proof}

\subsection{Proving an asymptotic lower bound}

We are now ready to prove \cref{lem:measure-change-approx}.
\begin{proof}[Proof of \cref{lem:measure-change-approx}]
    Firstly
    \begin{equation*}
        \prod_{i=1}^m \frac{(n-i+1)\mu}{\sum_{k=i}^m k^-_i + \Xi^-_{n-m}} = \exp(X_n - Y_n)
    \end{equation*}
    where
    \begin{equation*}
        X_n = \sum_{i=1}^m \log\left( 1 - \frac{i-1}{n} \right)
        \quad \text{and} \quad
        Y_n = \sum_{i=1}^m \log\left( \frac{\sum_{k=i}^m k_i^- + \Xi^-_{n-m}}{\mu n} \right).
    \end{equation*}
    Note
    \begin{equation*}
        \sum_{j=i}^m k^-_j = s^-(m) - s^-(i - 1) + (m - i + 1) \nu_-
    \end{equation*}
    and define
    \begin{equation*}
        \Omega^-_n \defeq \Xi^-_{n-m} - (n-m)\mu + (\nu_- + \mu)m.
    \end{equation*}
    Then $\Omega_n^-$ is almost the centering of $\Xi_{n-m}^-$ under $\Pr_n$. We have
    \begin{align*}
        Y_n 
        &= \sum_{i=1}^m \log \left( \frac{s^-(m) - s^-(i-1) + (m - i + 1) \nu_- + \Omega^-_n + \mu n - \nu_- m}{\mu n} \right) \\
        &= \sum_{i=1}^m \log \left( 1 + A_n^i + B_n + C_n^i \right)
    \end{align*}
    where
    \begin{align*}
        A_n^i = - \frac{1}{\mu n} \left[ s^-(i-1) - s^-(m) \right], \quad
        B_n = \frac{1}{\mu n} \Omega^-_n, \quad
        C_n^i = - \frac{\nu_-}{\mu n} (i-1).
    \end{align*}
    When expanding $\log(1 + A_n^i + B_n + C_n^i)$, the summation contributes order $m = O(n^{2/3})$. Thus we keep terms of order $\Omega(n^{-2/3})$ in the expansion. Write
    \begin{equation*}
        \omegaevent_n \defeq \left\{ \abs{\Omega^-_n} \leq 2n^{\frac{1}{2} + \epsilon} \right\}
    \end{equation*}
    On the event $\omegaevent_n$, we can check that the $A_n, B_n, C_n^i$ and $(C_n^i)^2$ terms are the only ones in the expansion which have order $\Omega(n^{-2/3})$. Moreover
    \begin{equation*}
        \sum_{i=1}^m C_n^i = - \frac{\nu_-}{2 \mu} \frac{m^2}{n} + \littleo(1) \quad \text{and} \quad
        \sum_{i=1}^m (C_n^i)^2 = \frac{\nu_-^2}{3 \mu^2} \frac{m^3}{n^2} + \littleo(1).
    \end{equation*}
    Therefore
    \begin{align*}
        Y_n
        &= \sum_{i=1}^m (A_n^i + B_n + C_n^i - \tfrac{1}{2} (C_n^i)^2) + \littleo(1) \\
        &= - \frac{1}{\mu} \frac{1}{n} \sum_{i=0}^m \left( s^-(i) - s^-(m) \right)
        + \frac{1}{\mu} \frac{m}{n} \Omega_n^-
        - \frac{\nu_-}{2\mu} \frac{m^2}{n} - \frac{\nu_-^2}{6 \mu^2} \frac{m^3}{n^2} + \littleo(1),
    \end{align*}
    where we use that $\sum_{i=1}^m \left( s^-(i-1) - s^-(m) \right) = \sum_{i=0}^m \left( s^-(i) - s^-(m) \right)$.

    Similarly we can expand $X_n$ as
    \begin{equation*}
        X_n = - \frac{1}{2} \frac{m}{n} - \frac{1}{3} \frac{m^3}{n^2} + \littleo(1).
    \end{equation*}

    Thus
    \begin{align*}
        \prod_{i=1}^m \frac{(n-i+1)\mu}{\sum_{k=i}^m k^-_i + \Xi^-_{n-m}}
        &\geq \exp \Bigg( \frac{1}{\mu} \frac{1}{n} \sum_{i=1}^m (s^-(i) - s^-(m)) \nonumber \\
        &\hspace{4em} - \frac{1}{\mu} \frac{m}{n} \Omega^-_n + \frac{\nu_- - \mu}{2 \mu} \frac{m^2}{n} + \frac{\nu_-^2 - \mu^2}{6 \mu^2} \frac{m^3}{n^2} \Bigg) \indi_{\omegaevent_n}
    \end{align*}
    In addition using \cref{lem:asym-expansions}, the measure change can be expanded as
    \begin{equation*}
        \diff{\Pr_n}{\Pr} = \exp \left( 
            - \frac{1}{\mu} \frac{m}{n} \Omega_n^- + \frac{\nu_- - \mu}{2\mu} \frac{m^2}{n}
            + \frac{\nu_-^2 - \mu^2}{6 \mu^2} \frac{m^3}{n^2} + \frac{\biasvar}{6 \mu^2} \frac{m^3}{n^2} + \littleo(1)
         \right).
    \end{equation*}
    Hence
    \begin{align*}
        &\conmc^n_m(\vect{k}_1, \ldots, \vect{k}_m) \\
        =& \frac{1}{\Pr(\Delta_n = 0)} \E \left[\prod_{i=1}^m \frac{(n-i+1)\mu}{\sum_{k=i}^m k^-_i + \Xi^-_{n-m}} \indi_{A_n} \right] \\
        \geq& \frac{1}{\Pr(\Delta_n = 0)} \E_n \left[ \exp \left(
                \frac{1}{\mu n} \sum_{i=1}^m \left( s^-(i) - s^-(m) \right)
                - \frac{\biasvar}{6 \mu^2} \frac{m^3}{n^2} + \littleo(1)
            \right) \indi_{\omegaevent_n \cap A_n} \right] \\
        \geq& \exp \left(
                \frac{1}{\mu n} \sum_{i=1}^m \left( s^-(i) - s^-(m) \right)
                - \frac{\biasvar}{6 \mu^2} \frac{m^3}{n^2}
            \right) (1 + \littleo(1)) \frac{\Pr_n(\omegaevent_n \cap A_n)}{\Pr(\Delta_n = 0)}.
    \end{align*}
    By \cref{lem:asym-expansions} we have that
    \begin{equation*}
        \Omega^-_n = \Xi^-_{n-m} - \E_n[\Xi^-_{n-m}] + \bigo(n^{1/3}).
    \end{equation*}
    In particular for all sufficiently large n,
    \begin{equation*}
        \Pr_n(\omegaevent_n \cap A_n) \geq
        \Pr_n\left(
            \abs{\Xi^-_{n-m} - \E[\Xi^-_{n-m}]} \leq n^{1/2 + \epsilon}, \Delta_{n-m} = \sum_{i=1}^m (k^+_i - k^-_i)
        \right).
    \end{equation*}
    Thus by \cref{lem:normal-llt} and \cref{lem:mod-dev-local},
    \begin{equation*}
        \frac{\Pr_n(\omegaevent_n \cap A_n)}{\Pr(\Delta_n = 0)} \geq 1 + \littleo(1)
    \end{equation*}
    as $n \to \infty$, which gives the desired final result.
\end{proof}

\section{Proof of measure change scaling limit}

We end on the proof of \cref{thm:measure-change}.

\begin{proof}[Proof of \cref{thm:measure-change}]
    The existence and exact form of the measure change are proven in \cref{lem:measure-change-exact-conditioned}. Let $m = \lfloor n^{2/3} t \rfloor$. Define
    \begin{equation*}
        S^{\pm}(k) = \sum_{i=1}^k (Z^{\pm}_i - \nu_{\pm})
    \end{equation*}
    and the event
    \begin{equation*}
        A_n = \left\{
            \max_{i=1, \ldots, m} \abs{S^-(i)} \leq m^{1/2 + \epsilon}, \ 
            \max_{i=1, \ldots, m} \abs{S^+(i)} \leq m^{1/2 + \epsilon}
        \right\}.
    \end{equation*}
    This is exactly the event that $\vZ_1, \ldots, \vZ_m$ satisfies the conditions of \cref{lem:measure-change-approx} and hence
    \begin{align*}
        \Phi(n, m)
        &= \phi^n_m( \vZ_1, \ldots, \vZ_m) \\
        &\geq \Bigg\{ \exp\left( \frac{1}{\mu n} \sum_{i=0}^m (S^-(i) - S^-(m)) - \frac{\biasvar}{6 \mu^2} \frac{m^3}{n^2} \right) + \littleo(1) \Bigg\} \indi_{A_n}.
    \end{align*}
    It is proven in \cite[Lemma 4.4]{conchon--kerjanStableGraphMetric2020} that
    \begin{equation*}
        \frac{1}{n} \sum_{i=0}^m (S^-(i) - S^-(m)) \todist \sigma_- \int_0^t (B(s) - B(t)) \dif s
    \end{equation*}
    where $B$ is a standard Brownian motion. Further $S^-(k)$ is a centered random walk and thus a martingale. Then $S^-(k)^2$ is a sub-martingale. Therefore by Doob's maximal inequality
    \begin{equation*}
        \Pr\left(
            \max_{i=1, \ldots, m} \abs{S^-(i)} > m^{1/2 + \epsilon}
        \right)
        \leq \frac{\E[(S^-(m))^2]}{m^{1 + 2 \epsilon}} = \frac{\sigma_-^2}{m^{2 \epsilon}} \to 0
    \end{equation*}
    as $n \to \infty$. The same argument applies to $S^+(k)$, thus $\Pr(A_n) \to 1$ as $n \to \infty$. Hence
    \begin{align*}
        &\Bigg\{ \exp\left( \frac{1}{\mu n} \sum_{i=0}^m (S^-(i) - S^-(m)) - \frac{\biasvar}{6 \mu^2} \frac{m^3}{n^2} \right) + \littleo(1) \Bigg\} \indi_{A_n}. \\
        \todist& \exp\left( \frac{\sigma_-}{\mu} \int_0^t (B(s) - B(t)) \dif s  - \frac{\biasvar}{6 \mu^2} t^3 \right) \\ 
        =& \exp\left( - \frac{\sigma_-}{\mu} \int_0^t s \dif B_s  - \frac{\biasvar}{6 \mu^2} t^3 \right).
    \end{align*}
    This has mean 1 and thus the final result follows by \cref{lem:lower-bound-suff}.
\end{proof}