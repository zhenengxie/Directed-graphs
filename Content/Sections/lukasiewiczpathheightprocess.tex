Let $T=(V,E,\rho)$ be an ordered rooted finite tree with vertex set $V$, edge set $E$ and root $\rho$; say $|V|=n$. Let $v_0,\dots,v_{n-1}$ denote the vertices of the tree visited in depth-first order, so that $v_0$ is the root of the tree. We can view $T$ as a metric space by regarding all edges as line segments of length $1$ that are connected via the vertices. The distance $d_T$ between points $a_1$ and $a_2$ on line segments $l_1$ and $l_2$ respectively is then defined as the length of the unique non-self-intersecting path between $a_1$ and $a_2$ that traverses the line segments of the tree. Denote $(T,d_T)$ by $\mathrm{T}$.\\
We will define the height function and \L ukasiewicz path of $\mathrm{T}$. Both of these functions uniquely characterize $\mathrm{T}$. The height process of $\mathrm{T}$, referred to as $h$, is defined as $$h(i)=d_T(v_i,v_0),$$ i.e.  for all $i$, $h(i)$ equals the distance from $v_i$ to the root.
Moreover, for all $i=1,\dots,n$, let $y_i$ be the number of children of $v_{i-1}$, and set $y_0=1$. Then, the \L ukasiewicz path of $\mathrm{T}$ is defined by $$s(i)=\sum\limits_{j\leq i} (y_j-1)$$ for $i=0,\dots,n$. Then, $s(i)$ is the total number of younger siblings of $v_i$ and its ancestors.

Suppose we have a function $\ell:E\to [0,\infty)$. Then, we can view $T$ as a metric space by regarding an $e$ as a line segment with length $\ell(e)$. The metric $d_T^\ell$ is defined similarly to $d_T$, and we denote the resulting metric space $(T,d^\ell_T)$ by $\mathrm{T}^\ell$, and call it a \emph{ordered rooted finite tree with edge lengths}. This gives rise to an alternative height process, referred to as $h^\ell$, which is defined $$h^\ell(i)=d^\ell_T(v_i,v_0),$$ i.e.  for all $i$, $h^\ell(i)$ equals the distance from $v_i$ to the root in $\mathrm{T}^\ell$. We set the \L ukasiewicz path of $\mathrm{T}^\ell$ equal to the \L ukasiewicz path of $\mathrm{T}$.

For a sequence of ordered rooted finite trees (with or without edge lengths), we define its height process by concatenating the height processes of the trees in the sequence. The \L ukasiewicz path is defined similarly. 