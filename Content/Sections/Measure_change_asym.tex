\subsection{Asymptotic lower bound on the measure change}

The following result, which is an analogue of \cite[Lemma 6.7]{conchon--kerjanStableGraphMetric2020}, describes the asymptotic behaviour of the measure change when $m = \Theta(n^{2/3})$.
\begin{lemma}
    \label{lem:measure-change-approx}
    Define
    \begin{equation*}
        s^{\pm}(i) = \textstyle{\sum_{j=1}^i (k_i^{\pm} - \lambda_{\pm})}.
    \end{equation*}
    Suppose that $\vk_1, \ldots, \vk_m$ are such that
    \begin{equation}
        \label{eq:s-condition}
        \max_{i=1, \ldots, m} \abs{s^{-}(i)} \leq m^{\frac{1}{2}} \log(m)
        \quad \text{and} \quad
        \max_{i=1, \ldots, m} \abs{s^{+}(i)} \leq m^{\frac{1}{2}} \log(m)
    \end{equation}
    Then in the regime $m = \Theta(n^{2/3})$,
    \begin{equation*}
        \phi^n_m(\vk_1, \ldots, \vk_m)
        \geq \exp\left( \frac{1}{\mu n} \sum_{i=0}^m (s^-(i) - s^-(m)) - \frac{\sigma_-}{6 \mu^2} \frac{m^3}{n^2} \right) + \littleo(1),
    \end{equation*}
    where the $\littleo(1)$ term is independent of $\vk_1, \ldots, \vk_m$ satisfying our assumptions.
\end{lemma}
Let us explain the condition in \cref{eq:s-condition}. In \cref{thm:measure-change}, we evaluate $\phi^n_m$ as 
\begin{equation*}
    \phi^n_m(\vZ_1, \ldots, \vZ_m).
\end{equation*}
Thus the condition in \cref{eq:s-condition} corresponds to the event
\begin{equation*}
    \max_{i=1, \ldots, m } \abs*{
        \sum_{j=1}^i (Z^{-}_j - \lambda_{-}) 
    } \leq m^{1/2} \log(m)
    \quad \text{and} \quad
    \max_{i=1, \ldots, m } \abs*{
        \sum_{j=1}^i (Z^{+}_j - \lambda_{+}) 
    } \leq m^{1/2} \log(m)
\end{equation*}
This says that the centered random walks corresponding to $Z^+_i$ and $Z^-_i$ do not deviate by more than $m^{1/2} \log(m)$ in the first $m$ steps. This event will occur with high probability, and so \cref{eq:s-condition} is not a restrictive condition to take.

The fact that we only prove a lower bound may seem strange at first. First note that all measure changes are non-negative random variables and have expectation 1. Hence if the sequence of lower bounds on the measure changes converge to a limit that also has expectation 1, then we have not have lost a significant amount of probability mass and thus the measure changes converge to the same limit as the lower bounds. The following lemma, adapted from \cite[Lemma 4.8]{conchon--kerjanStableGraphMetric2020}, makes this formal.
\begin{lemma}
    \label{lem:sandwiching-lemma}
    Let $(X_n, Y_n, Z_n)_{n \geq 1}$ be a sequence of $[0, \infty) \times [0, \infty) \times S$-valued random variables where $S$ is a metric space. Suppose there exists a $[0, \infty) \times S$-valued random variable $(Y, Z)$ such that the following holds:
    \begin{enumerate}
        \item $(Y_n, Z_n) \todist (Y, Z)$ as $n \to \infty$.
        \item $X_n \geq Y_n$ almost surely for all $n$.
        \item $\E[X_n] = 1$ for all $n$ and $\E[Y] = 1$.
    \end{enumerate}
    Then $(X_n, Z_n) \todist (Y, Z)$ also. Moreover $(X_n)_{n \geq 1}$ is a sequence of uniformly integrable random variables.
\end{lemma}
\begin{proof}
    We first show $X_n - Y_n \to 0$ in $L^1$ as $n \to \infty$. Since $X_n \geq Y_n$,
    \begin{equation*}
        \limsup_{n \to \infty} \E \abs{X_n - Y_n}
        = \limsup_{n \to \infty} \E[X_n - Y_n]
        = 1 - \liminf_{n \to \infty} \E[Y_n].
    \end{equation*}
    Thus it suffices to show $\liminf_n \E[Y_n] \geq 1$. This follows since
    \begin{equation*}
        \liminf_n \E[Y_n] =
        \liminf_n \E[\abs{Y_n}] \geq
        \E[\abs{Y}] =
        \E[Y] = 1,
    \end{equation*}
    where the first and second equalities follow from the non-negativity of $Y_n$ and $Y$, and the inequality follows by the Portmanteau theorem.

    In particular $X_n - Y_n \todist 0$. Since this limit is a constant, it follows by Slutsky's theorem that
    \begin{equation*}
        \vectwo{X_n}{Z_n} = \vectwo{Y_n}{Z_n} + \vectwo{X_n - Y_n}{0} \todist \vectwo{Y}{Z}.
    \end{equation*}

    We now prove the uniform integrability. By Skorokhod's representation theorem we may suppose that $X_n \to Y$ almost surely. Then $\E[X_n] = \E[Y] = 1$ for all $n$. Therefore, $X_n \to Y$ in $L^1$ by Scheffé's lemma, which entails the required uniform integrability.
\end{proof}

\subsubsection{Local Limit Theorems}

The denominator of $\phi^n_m$, as given by \cref{lem:exact-measure-change}, is $\P(\Delta_n = 0)$. The random variable $\Delta_n$ is a sum of independent random variables and the asymptotic behaviour of such a sum being equal to some value is described by the local limit theorem.

Let $X_1, X_2, \ldots$ be i.i.d.\ random variables with mean $\mu$ and finite variance $\sigma^2$. Then, by the central limit theorem,
\begin{equation*}
    \tfrac{1}{\sqrt{n}} \big( \textstyle \sum_{i=1}^n X_i - n \mu \big) \todist N(0, \sigma^2)
\end{equation*}
as $n \to \infty$. Hence we conjecture that
\begin{equation*}
    f(X) = \sin(X) + 1
\end{equation*}
where $f$ is the density of a standard normal distribution.

\subsubsection{Asymptotic behaviour of the numerator of the measure change}

The numerator takes form OIJOIJOJ and we are interested in the case where INSERT CONDITION. There are currently two problems.

Firstly note that
\begin{equation*}
    \E[Z^- - Z^+] = \frac{1}{\mu} \E[D^-D^+ - (D^-)^2].
\end{equation*}
This quantity is, in general, non-zero, even if $\E[D^- - D^+] = 0$. The deviation of a sum of the $Z^{\pm}_i$ around its mean is controlled by the assumption in \cref{eq:s-condition}. If $\vk_1, \ldots, \vk_n$ satisfy \cref{eq:s-condition} and $m = \Theta(n^{2/3})$ then
\begin{equation*}
    \sum_{i=1}^m (k_i^- - k_i^+) = s^-(m) - s^+(m) + (\lambda_+ - \lambda_-) m = \Theta(n^{2/3}).
\end{equation*}

In contrast, $\Delta_{n-m}$ is centered, so
\begin{equation*}
    \left\{ \Delta_{n-m} = \textstyle \sum_{i=1}^m (k_i^+ - k_i^-) \right\}  
\end{equation*}
is looking at the event that $\Delta_{n-m}$ takes a value at distance $n^{2/3}$ away from its mean. As remarked earlier INSERT REFERENCE, the local limit provides no useful information in this regime. 

The second problem is that even in absence of the indicator function, the expectation of the exponential of a random variable is often not dictated by the standard behaviour of the random variable. Although large values of IJOIJ are less likely, the exponential FINISH EXPLANATION.

We address both of these issues by means introducing a sequence of exponentially tilted measures.

The effect of the exponentially tilted measures will be to shift the mean of $\Delta_{n-m}$ in such a way that, under the new measure, the event concerns only a typical deviation. The next result defines this tilt and then gives asymptotic expansions for cumulant generating function of $D^-$, the mean of $D^-$ and the mean of $D^+$ under this tilting. 
\begin{lemma}
    \label{lem:asym-expansions}
    Define an measure $\P_{\theta}$, for $\theta \geq 0$, by its Radon--Nikodym derivative
    \begin{equation*}
        \diff{\P_{\theta}}{\P} = \exp\left( - \theta D^- - \alpha(\theta) \right)
        \quad \text{where} \quad
        \alpha(\theta) = \log \E \left[ e^{-\theta D^-} \right].
    \end{equation*}
    Then as $\theta \downarrow 0$ we have
    \begin{align*}
        \alpha(\theta) &= -\mu \theta + \tfrac{1}{2}\var(D^-) \theta^2 - \tfrac{1}{6} \E \left[ (D^- - \mu)^3 \right] \theta^3 + \littleo(\theta^3), \\
        \E_{\theta}[D^-] &= \mu - \var(D^-) \theta + \bigo(\theta^2), \\
        \text{and} \quad \E_{\theta}[D^+] &= \mu - \cov(D^-, D^+) \theta + \bigo(\theta^2).
    \end{align*}
\end{lemma}
\begin{proof}
    Since $\E\left[\abs{D^-}^3\right] < \infty$ and $D^-$ is non-negative, by the dominated convergence theorem
    \begin{equation}
        \E \left[ (D^-)^3 \exp(-\theta D^-) \right] = \E \left[ (D^-)^3 \right] + \littleo(1)
        \label{eq:mgf-start}
    \end{equation}
    as $\theta \downarrow 0$. Integrating \cref{eq:mgf-start} with respect to $\theta$ and applying Fubini's theorem to exchange the order of the expectation and integral gives
    \begin{equation*}
        \E \left[ \int_0^{\theta} (D^-)^3 e^{-\theta' D^-} \dif \theta' \right]
        = \E \left[ \int_0^{\theta} \left\{ (D^-)^3 + \littleo(1) \right\} \dif \theta' \right]
        = \E \left[ (D^-)^3 \right] \theta + \littleo(\theta).
    \end{equation*}
    Evaluating the integral with respect to $\theta'$ on the left hand side and rearranging gives that
    \begin{equation*}
        \E \left[ (D^-)^2 e^{-\theta D^-} \right]
        = \E \left[ (D^-)^2 \right] - \E \left[ (D^-)^3 \right] \theta + \littleo(\theta).
    \end{equation*}
    Repeating this method yields
    \begin{align}
        \E \left[ D^- e^{-\theta D^-} \right]
        &= \mu - \E \left[ (D^-)^2 \right] \theta + \tfrac{1}{2} \E \left[ (D^-)^3 \right] \theta^2 + \littleo(\theta^2), \label{eq:asymin} \\
        \text{and} \quad \E \left[ e^{-\theta D^-} \right]
        &= 1 - \mu \theta + \tfrac{1}{2} \E \left[ (D^-)^2 \right] \theta^2 - \tfrac{1}{6} \E \left[ (D^-)^3 \right] \theta^3 + \littleo(\theta^3).
        \label{eq:asym00}
    \end{align}
    Similarly integrating the equation
    \begin{equation*}
        \E \left[ (D^-)^2 D^+ \exp(-\theta D^-) \right] = \E \left[ (D^-)^2 D^+ \right] + \littleo(1)
    \end{equation*}
    twice gives
    \begin{equation}
        \E \left[ D^+ e^{-\theta D^-} \right]
        = \mu \theta - \E \left[ D^- D^+ \right] \theta + \tfrac{1}{2} \E \left[ (D^-)^2 D^+ \right] \theta^2 + \littleo(\theta^2).
        \label{eq:asymout}
    \end{equation}
    \cref{eq:asym00} gives the small-$\theta$ expansion of the normalising constant of the measure change. Combining this with \cref{eq:asymin} and \cref{eq:asymout} yields the expansions for $\E_{\theta}[D^-]$ and $\E_{\theta}[D^+]$ respectively. Taking the logarithm of \cref{eq:asym00} gives the expansion of the cumulant generating function $\alpha(\theta)$.
\end{proof}

To achieve the recentering of $\Delta_{n-m}$ we desire, let us define a sequence of tilted measures $\P_n$ defined by their Radon--Nikodym derivative
\begin{equation}
    \diff{\P_n}{\P} = \exp \left( - \theta_n \Xi^-_{n-m} - (n - m) \alpha(\theta_n) \right),
\end{equation}
where $\theta_n = \frac{m}{\mu n}$. This factorises and so $\vD_1, \ldots, \vD_n$ remain i.i.d.\ under this tilting, each having the law of $\vD$ under $\P_{\theta_n}$. Applying \cref{lem:asym-expansions}, we can compute that
\begin{align*}
    \E_n[\Delta_{n-m}] 
    &= m(\lambda_+ - \lambda_-) + \bigo(n^{1/3}).
\end{align*}
Hence,
\begin{align*}
    \textstyle \sum_{i=1}^m (k_i^+ - k_i^-) - \E_n[\Delta_{n-m}] 
    &= s^-(m) - s^+(m) + \Big[ m(\lambda_+ - \lambda_-) - \E_n[\Delta_{n-m}] \Big] \\
    &= \bigo(n^{1/3 + \epsilon}),
\end{align*}
which is within the $\bigo(n^{1/2})$ range from the mean required for a typical deviation. This justifies our choice of $\theta_n = \frac{m}{\mu n}$.

It will turn out that the same exponential tilting is approximately achieved by the Radon-Nikodym derivative $\phi^n_m(\vk_1, \ldots, \vk_m)$, and so this perspective will play a key role in the proof of \cref{lem:measure-change-approx}.

\subsubsection{Multivariate Local Limit Theorem}

We wish to determine the local limit behaviour of $\Delta_n$ under the untilted measure and $\Delta_{n-m}$ under the tilted measures. In the latter case, we wish to show the behaviour remains the same even when conditioning on $\Xi_{n-m}^-$ having fluctuations of order $\bigo(n^{1/2 + \epsilon})$ about its tilted mean. The way we show this is to first show a bivariate local limit theorem. Such a theorem is described by Muk

To make this work we first need to define the analogue of span for $\Z^2$-valued random variables.

$(D^- - D^+, D^-)$ is $\Z^2$ valued and thus certainly lattice valued. Further, in the degenerate case where $D^- = D^+$ almost surely, the total in-degree and total out-degree will be equal almost surely and thus the conditioned and non-conditioned case are the same. So the proof of \cref{thm:measure-change} in that case is the same as that of \cite[Proposition 4.3]{conchon--kerjanStableGraphMetric2020}. Hence WLOG assume $(D^- - D^+, D^-)$ is non-degenerate. Let $\lattice$ be the main lattice of $(D^- - D^+, D^-)$. Then by \cref{lem:zd-triangular} there exist $p, q, r \in \Z_{\geq 0}$ such that $\lattice$ is generated by the columns of
\begin{equation*}
    \begin{pmatrix}
        p & 0 \\
        r & q
    \end{pmatrix}
\end{equation*}
We have assumed that $D^- - D^+$ has main lattice $\Z$ and therefore $p = 1$. Conditional on $D^- - D^+$ taking some value, $D^-$ will have main lattice $q \Z$. Let $\sigma^2$ be the variance of $D^- - D^+$ and $\Sigma$ be the covariance matrix of $(D^- - D^+, D^-)$.

We will show that $\P(R_n < m)$ decays exponentially and so the asymptotic behaviour of $\P(R_n \geq m, \Delta_n = 0)$ is dictated by the $\Delta_n = 0$ event. This can be handled by the standard local limit theorem.
\begin{lemma}
    \label{lem:normal-llt}
    As $n \to \infty$
    \begin{equation*}
        \P(R_n \geq m, \Delta_n = 0) = \frac{1}{\sqrt{2 \pi \sigma^2 n}} \left( 1 + \littleo(1) \right).
    \end{equation*}
\end{lemma}
\begin{proof}
    $R_n \sim \text{Binomial}(n, p)$ for $p > 0$ and, for sufficiently large $n$, $m \leq \frac{1}{2} pn$. So by Hoeffding's inequality,
    \begin{equation*}
        \P(\Delta_n = 0) - \P(R_n \geq m, \Delta_n = 0) \leq \P(R_n < m) \leq \P(\abs{R_n - np} < \tfrac{1}{2}pn) \leq e^{-cn}
    \end{equation*}
    for some $c > 0$.  The random variable $D^- - D^+$ is assumed to have finite variance $\sigma$ and main lattice $\Z$. Moreover it is centered and $0$ is in the main lattice $\Z$. Thus,
    \begin{equation*}
        \P(R_n \geq m, \Delta_n = 0) = \P(\Delta_n = 0) + \littleo(n^{-1/2}) = \frac{1}{\sqrt{2 \pi \sigma^2 n}}(1 + \littleo(1)),
    \end{equation*}
    where the second equality is by \cref{thm:normal-multi-llt}.
\end{proof}

Next we show the local limit theorem holds for $(\Delta_{n-m}, \Xi^-_{n-m})$ under the tilting.
\begin{lemma}
    \label{lem:bivar-llt}
    Let $\vc_n \in \Z^2$ be such that $\vc_n + \Lambda$ contains the support of $(\Delta_{n-m}, \Xi^-_{n-m})$. Then uniformly for $(x, y) \in \vc_n + \Lambda$,
    \begin{align*}
        &\P_n\left(
            \Delta_{n-m} = \E \big[ \Delta_{n-m} \big] + x, \ 
            \Xi^-_{n-m} = \E \big[ \Xi^-_{n-m} \big] + y
        \right)  \nonumber \\
        &\hspace{7em} = \frac{q}{2\pi \det(\Sigma)^{1/2} \: n} \exp\left( 
            \frac{-1}{2n}
            \begin{pmatrix}
                x & y
            \end{pmatrix}
            \Sigma
            \begin{pmatrix}
                x \\ y
            \end{pmatrix}
         \right)
         + \littleo( n^{-1} )
    \end{align*}
    as $n \to \infty$.
\end{lemma}
\begin{proof}
    Let $\vX_{n1}, \ldots, \vX_{nn}$ have the same distribution as 
    \begin{equation*}
        \vectwo{D^-_1 - D^+_1}{D^-_1}, \ldots, \vectwo{D^-_n - D^+_n}{D^-_n}
    \end{equation*}
    under the tilted measure $\P_n$. Then since $\theta_n = \littleo(1)$, it is simple to show that $\vX_{n1}$ tends weakly in law to $\vX = (D^- - D^+, D^-)$ under the non-tilted measure. The measure change does not change the support of the random variables and thus all $\vX_{n1}$ and $\vX$ have the same main lattice $\lattice$. Finally, we check the uniform integrability condition. Since $\theta_n = \littleo(1)$, we have $M = - \inf_n \alpha(\theta_n) < \infty$. Then
    \begin{align*}
        \sup_n \E[\norm{\vX_{n1}}^2 \one \{ \norm{\vX_{n1}}^2 \geq L \} ]
        &= \sup_n \E[e^{-\theta_n D^- - \alpha(\theta_n)} \norm{\vX}^2 \one\{\norm{\vX}^2 \geq L \} ] \\
        &\leq e^M \E[\norm{\vX}^2 \one\{\norm{\vX}^2 \geq L \} ] \\
        & \to 0
    \end{align*}
    as $L \to \infty$, since $\vX$ is assumed to have finite second moment. Thus the desired result follows from \cref{thm:multi-triangular-llt}. There is a slight change in that we have $n - m$ terms in the summations for $\Delta_{n-m}$ and $\Xi^-_{n-m}$ rather than $n$ terms, but since $m = \Theta(n^{2/3})$ this does not matter asymptotically.
\end{proof}

Now we show the $\P(\Delta_{n-m} = \sum_{i=1}^m (k_i^+ - k_i^-))$ have the same asymptotic behaviour as $\P(\Delta_n = 0)$ even when we condition on $\Xi_{n-m}^-$ not `varying too much' about its tilted mean. We only prove a lower bound, but this is sufficient for proving \cref{lem:measure-change-approx}.
\begin{lemma}
    \label{lem:mod-dev-local}
    Under the assumptions of \cref{lem:measure-change-approx},
    \begin{equation*}
        \mathbb P_n \left(
            \Delta_{n-m} = \sum_{i=1}^m (k_i^+ - k_i^-), \ 
            \abs{\Xi^-_{n-m} - \E_n[\Xi^-_{n-m}]} \leq n^{\frac{1}{2} + \epsilon}
        \right)
        \geq \frac{1}{\sqrt{2 \pi \sigma^2 n}} (1 + \littleo(1)).
    \end{equation*}
\end{lemma}
\begin{proof}
    For convenience let
    \begin{equation*}
        P_n = \mathbb P_n \left(
            \Delta_{n-m} = \sum_{i=1}^m (k_i^+ - k_i^-), \ 
            \abs{\Xi^-_{n-m} - \E_n[\Xi^-_{n-m}]} \leq n^{\frac{1}{2} + \epsilon}
        \right).
    \end{equation*}
    Define
    \begin{equation*}
        a_n = \sum_{i=1}^m (k_i^+ - k_i^-) - \E_n[\Delta_{n-m}].
    \end{equation*}
    Also let
    \begin{equation*}
        L_n = \Big\{
            y : \bigg( \textstyle\sum_{i=1}^m (k_i^+ - k_i^-), y \bigg) \in \vc_n + \lattice
            \Big\}.
    \end{equation*}
    $L_n$ has a simpler representation. Fix any $y_0 \in L_n$. Then if $\lattice$ is generated by the columns of
    \begin{equation*}
        \begin{pmatrix}
            1 & 0 \\
            r & q
        \end{pmatrix}
    \end{equation*}
    we must have $L_n = y_0 + q\Z$. Fix an arbitrary $M > 0$. Then
    \begin{align*}
        P_n &= \sum_{\substack{y \in L_n \\ \abs{y} \leq n^{1/2 + \epsilon}}} \mathbb P_n \left( \Delta_{n-m} = \E_n[\Delta_{n-m}] + a_n, \ \Xi^-_{n-m} = \E_n[\Xi^-_{n-m}] + y \right) \\
        &\geq \sum_{\substack{y \in L_n \\ \abs{y} \leq M n^{1/2}}} \mathbb P_n \left( \Delta_{n-m} = \E_n[\Delta_{n-m}] + a_n, \ \Xi^-_{n-m} = \E_n[\Xi^-_{n-m}] + y \right)
    \end{align*}
    for all $n$ sufficiently large. By \cref{lem:bivar-llt}, using that the error is uniform, we have that
    \begin{equation*}
        P_n \geq \sum_{\substack{y \in L_n \\ \abs{y} \leq M n^{1/2}}} 
         \frac{q}{2\pi \det(\Sigma)^{1/2} \: n} \exp\left( 
            \frac{-1}{2n} \vectwo{a_n}{y} \cdot \Sigma^{-1} \vectwo{a_n}{y}
         \right)
         + \littleo( n^{-1/2} )
    \end{equation*}

    We wish to factorise the summand. To this end, we make a change of variables. There exists $c \in \R$ such that
    \begin{equation*}
        \cov(D^- - c(D^- - D^+), D^- - D^+) = 0.
    \end{equation*}
    Let $\tau^2$ be the variance of $D^- - c(D^- - D^+)$. Then
    \begin{align*}
         &\frac{q}{2\pi \det(\Sigma)^{1/2} \: n} \exp\left( 
            \frac{1}{2n} \vectwo{a_n}{y} \cdot \Sigma^{-1} \vectwo{a_n}{y}
         \right) \nonumber \\
         & \hspace{5em} =
         \frac{1}{\sqrt{2 \pi \sigma^2 n}} \exp \left( - \frac{1}{2 \sigma^2} \frac{a_n^2}{n} \right)
         \frac{q}{\sqrt{2 \pi \tau^2 n}} \exp \left( - \frac{1}{2 \tau^2} \frac{(y - ca_n)^2}{n} \right).
    \end{align*}
    We now examine the asymptotic behaviour of $a_n$. By \cref{lem:asym-expansions},
    \begin{align*}
        \E_n[\Delta_{n-m}]
        &= (n - m) \E_{\theta_n}[D^- - D^+] \\ 
        &= -(\lambda_- - \lambda_+)m + \bigo(n^{1/3}).
    \end{align*}
    Therefore 
    \begin{equation*}
        a_n = s_+(m) - s_-(m) + \bigo(n^{1/3}) = \bigo(n^{1/3 + \epsilon}),
    \end{equation*}
    by the assumption in \cref{eq:s-condition}. Thus
    \begin{equation*}
        P_n \geq
        \frac{1}{\sqrt{2 \pi \sigma^2 n}}(1 + \littleo(1))
        \sum_{\substack{y \in L_n \\ \abs{y} \leq M n^{1/2}}}
        \frac{q}{\sqrt{2 \pi \tau^2 n}} \exp \left( - \frac{1}{2 \tau^2} \frac{(y - ca_n)^2}{n} \right)
        + \littleo(n^{-1/2})
    \end{equation*}
    Note that
    \begin{equation*}
        \sum_{\substack{y \in L_n \\ \abs{y} \leq M n^{1/2}}}
        \frac{q}{\sqrt{2 \pi \tau^2 n}} \exp \left( - \frac{1}{2 \tau^2} \frac{(y - ca_n)^2}{n} \right)
        = \sum_{\substack{y \in L_n \\ \abs{y} \leq M n^{1/2}}}
        \frac{q}{\sqrt{n}}\ g\left( \frac{y - ca_n}{\sqrt{n}} \right)
    \end{equation*}
    where
    \begin{equation*}
        g(z) = \frac{1}{\sqrt{2 \pi \tau^2}} \exp\left( \frac{-z^2}{2 \tau^2} \right).
    \end{equation*}
    Since $a_n = \bigo(n^{1/3 + \epsilon})$, for $n$ sufficiently large
    \begin{align}
        \sum_{\substack{y \in L_n \\ \abs{y} \leq M n^{1/2}}}
        \frac{q}{\sqrt{2 \pi \tau^2 n}} \exp \left( - \frac{1}{2 \tau^2} \frac{(y - ca_n)^2}{n} \right)
        &\geq \sum_{\substack{z \in L_n - ca_n\\ \abs{z} \leq \frac{1}{2} M n^{1/2}}} 
        \frac{q}{\sqrt{n}}\ g \left( \frac{z}{\sqrt{n}} \right) \\
        &= \sum_{\substack{z \in \tilde{L}_n \\ \abs{z} \leq \frac{1}{2} M }}
        \frac{q}{\sqrt{n}}\ g(z) \label{eq:riemann-sum}
    \end{align}
    where
    \begin{equation*}
        \tilde{L}_n = \frac{L_n - ca_n}{\sqrt{n}}.
    \end{equation*}
    Then $\tilde{L}_n \cap [-\frac{1}{2}M, \frac{1}{2}M]$ is a partition of $[-\frac{1}{2}M, \frac{1}{2}M]$ where adjacent points are distance $q/\sqrt{n}$ apart from each other. Thus \cref{eq:riemann-sum} is a Riemann sum approximation of an integral. Hence
    \begin{equation*}
        \sum_{\substack{y \in L_n \\ \abs{y} \leq M n^{1/2}}}
        \frac{q}{\sqrt{2 \pi \tau^2 n}} \exp \left( - \frac{1}{2 \tau^2} \frac{(y - ca_n)^2}{n} \right)
        \geq (1 + \littleo(1)) \int_{-\frac{1}{2} M}^{\frac{1}{2}M} g(z) \dif z.
    \end{equation*}
    Thus
    \begin{equation*}
        P_n \geq \frac{1}{\sqrt{2 \pi \sigma^2 n}} (1 + \littleo(1)) \int_{-\frac{1}{2} M}^{\frac{1}{2}M} g(z) \dif z.
    \end{equation*}
    This holds for all $M > 0$, and $\int_{-\infty}^{\infty} g(z) \dif z = 1$. Therefore,
    \begin{equation*}
        P_n \geq \frac{1}{\sqrt{2 \pi \sigma^2 n}} \left( 1 + \littleo(1) \right),
    \end{equation*}
    as required.
\end{proof}

\subsubsection{Poof of the asymptotic lower bound}

We are now ready to prove \cref{lem:measure-change-approx}.
\begin{proof}[Proof of \cref{lem:measure-change-approx}]
    Firstly,
    \begin{equation*}
        \prod_{i=1}^m \frac{(n-i+1)\mu}{\sum_{k=i}^m k^-_i + \Xi^-_{n-m}} = \exp(X_n - Y_n),
    \end{equation*}
    where
    \begin{equation*}
        X_n = \sum_{i=1}^m \log\left( 1 - \frac{i-1}{n} \right)
        \quad \text{and} \quad
        Y_n = \sum_{i=1}^m \log\left( \frac{\sum_{k=i}^m k_i^- + \Xi^-_{n-m}}{\mu n} \right).
    \end{equation*}
    Note that
    \begin{equation*}
        \sum_{j=i}^m k^-_j = s^-(m) - s^-(i - 1) + (m - i + 1) \lambda_-,
    \end{equation*}
    and define
    \begin{equation*}
        \Omega^-_n = \Xi^-_{n-m} - (n-m)\mu + (\lambda_- + \mu)m.
    \end{equation*}
    Then $\Omega_n^-$ is almost the centering of $\Xi_{n-m}^-$ under $\P_n$. We have
    \begin{align*}
        Y_n 
        &= \sum_{i=1}^m \log \left( \frac{s^-(m) - s^-(i-1) + (m - i + 1) \lambda_- + \Omega^-_n + \mu n - \lambda_- m}{\mu n} \right) \\
        &= \sum_{i=1}^m \log \left( 1 + A_n^i + B_n + C_n^i \right)
    \end{align*}
    where
    \begin{align*}
        A_n^i = - \frac{1}{\mu n} \left[ s^-(i-1) - s^-(m) \right], \quad
        B_n = \frac{1}{\mu n} \Omega^-_n, \quad
        C_n^i = - \frac{\lambda_-}{\mu n} (i-1).
    \end{align*}
    The summation contributes order $m = O(n^{2/3})$. Thus we keep terms of order $\Omega(n^{-2/3})$ when expanding $\log(1 + A_n^i + B_n + C_n^i)$. Write
    \begin{equation*}
        \omegaevent_n = \left\{ \abs{\Omega^-_n} \leq 2n^{\frac{1}{2} + \epsilon} \right\}.
    \end{equation*}
    On the event $\omegaevent_n$, we can check that the $A_n^i, B_n, C_n^i$ and $(C_n^i)^2$ terms are the only ones in the expansion which have order $\Omega(n^{-2/3})$. Moreover,
    \begin{equation*}
        \sum_{i=1}^m C_n^i = - \frac{\lambda_-}{2 \mu} \frac{m^2}{n} + \littleo(1) \quad \text{and} \quad
        \sum_{i=1}^m (C_n^i)^2 = \frac{\lambda_-^2}{3 \mu^2} \frac{m^3}{n^2} + \littleo(1).
    \end{equation*}
    Therefore,
    \begin{align*}
        Y_n
        &= \sum_{i=1}^m (A_n^i + B_n + C_n^i - \tfrac{1}{2} (C_n^i)^2) + \littleo(1) \\
        &= - \frac{1}{\mu n} \sum_{i=0}^m \left( s^-(i) - s^-(m) \right)
        + \frac{m}{\mu n} \Omega_n^-
        - \frac{\lambda_-}{2\mu} \frac{m^2}{n} - \frac{\lambda_-^2}{6 \mu^2} \frac{m^3}{n^2} + \littleo(1),
    \end{align*}
    where we use that $\sum_{i=1}^m \left( s^-(i-1) - s^-(m) \right) = \sum_{i=0}^m \left( s^-(i) - s^-(m) \right)$.

    Similarly we can expand $X_n$ as
    \begin{equation*}
        X_n = - \frac{m}{2 n} - \frac{m^3}{3 n^2} + \littleo(1).
    \end{equation*}

    Thus,
    \begin{align*}
        \prod_{i=1}^m \frac{(n-i+1)\mu}{\sum_{k=i}^m k^-_i + \Xi^-_{n-m}}
        &\geq \exp \Bigg( \frac{1}{\mu n} \sum_{i=1}^m (s^-(i) - s^-(m)) \nonumber \\
        &\hspace{4em} - \frac{m}{\mu n} \Omega^-_n + \frac{(\lambda_- - \mu)}{2 \mu} \frac{m^2}{n} + \frac{(\lambda_-^2 - \mu^2)}{6 \mu^2} \frac{m^3}{n^2} \Bigg) \one_{\omegaevent_n}.
    \end{align*}
    In addition, using \cref{lem:asym-expansions}, the measure change can be expanded as
    \begin{equation*}
        \diff{\P_n}{\P} = \exp \left( 
            - \frac{m}{\mu n} \Omega_n^- + \frac{(\lambda_- - \mu)}{2\mu} \frac{m^2}{n}
            + \frac{(\lambda_-^2 - \mu^2)}{6 \mu^2} \frac{m^3}{n^2} + \frac{\sigma_-}{6 \mu^2} \frac{m^3}{n^2} + \littleo(1)
         \right).
    \end{equation*}
    Hence,
    \begin{align*}
        &\phi^n_m(\vk_1, \ldots, \vk_m) \\
        =& \frac{1}{\P(R_m \geq n, \Delta_n = 0)} \E \left[\prod_{i=1}^m \frac{(n-i+1)\mu}{\sum_{k=i}^m k^-_i + \Xi^-_{n-m}} \one_{A_n} \right] \\
        \geq& \frac{1}{\P(R_n \geq m, \Delta_n = 0)} \E_n \left[ \exp \left(
                \frac{1}{\mu n} \sum_{i=1}^m \left( s^-(i) - s^-(m) \right)
                - \frac{\sigma_-}{6 \mu^2} \frac{m^3}{n^2} + \littleo(1)
            \right) \one_{\omegaevent_n \cap A_n} \right] \\
        \geq& \exp \left(
                \frac{1}{\mu n} \sum_{i=1}^m \left( s^-(i) - s^-(m) \right)
                - \frac{\sigma_-}{6 \mu^2} \frac{m^3}{n^2}
            \right) (1 + \littleo(1)) \frac{\P_n(\omegaevent_n \cap A_n)}{\P(R_m \geq n, \Delta_n = 0)}.
    \end{align*}
    By \cref{lem:asym-expansions} we have that
    \begin{equation*}
        \Omega^-_n = \Xi^-_{n-m} - \E_n[\Xi^-_{n-m}] + \bigo(n^{1/3}).
    \end{equation*}
    In particular, for all sufficiently large $n$,
    \begin{equation*}
        \P_n(\omegaevent_n \cap A_n) \geq
        \P_n\left(
            \abs{\Xi^-_{n-m} - \E[\Xi^-_{n-m}]} \leq n^{1/2 + \epsilon}, \Delta_{n-m} = \sum_{i=1}^m (k^+_i - k^-_i)
        \right).
    \end{equation*}
    Thus by \cref{lem:normal-llt,lem:mod-dev-local},
    \begin{equation*}
        \frac{\P_n(\omegaevent_n \cap A_n)}{\P(R_n \geq m, \Delta_n = 0)} \geq 1 + \littleo(1)
    \end{equation*}
    as $n \to \infty$, which gives the desired final result.
\end{proof}