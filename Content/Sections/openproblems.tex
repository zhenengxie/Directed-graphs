\section{Open problems}\label{subsec.openproblems}
Our work contains the first quantitative results on the directed configuration model at criticality, and is the second metric space convergence result for a directed graph model (after the directed Erd\H{o}s-Rényi graph was studied in \cite{goldschmidtScalingLimitCritical2019}), and many interesting unresolved questions remain.
\begin{enumerate}
    \item The law of our limit object is defined by three parameters that are functions of the (mixed) moments of the degree distribution. Does a different choice of parameters always give a different limit distribution? If so, are the laws absolutely continuous to one another? 
    \item Our methods show that the diameter of the configuration model at criticality is  $\Omega(n^{1/3})$ in probability, which is in contrast with the off-critical cases (for deterministic degrees), in which the diameter is $\Theta(\log(n))$ in probability \cite{caiDiameterDirectedConfiguration2020}. We conjecture that the diameter is in fact $\Theta(n^{1/3})$ in probability. Goldschmidt and Maazoun are working on this question for the directed Erd\H{o}s-Rényi graph at criticality. 
    \item In \cite{goldschmidtScalingLimitCritical2019}, the authors show convergence of the sequence of SCCs in the $\ell_1$-sense, which is stronger than the product topology as considered by us. This for example implies that for the directed Erd\H{o}s-Rényi graph, under rescaling, the total length in the SCCs converges in distribution to some finite random variable. Also for undirected configuration models, there are no results that show metric space convergence in a topology on the sequence of components that is stronger than the product topology \cite{Bhamidi2020,conchon--kerjanStableGraphMetric2020,Bhamidi2020Glmb}.
     \item We conjecture that, just like the directed Erd\H{o}s-Rényi graph \cite{goldschmidtScalingLimitCritical2019}, the directed configuration model gives rise to a critical window, that in some sense interpolates between subcritical and supercritical models. It would be interesting to adapt our methods to the critical window.
     \item In future work, we plan to extend our understanding of the SCCs by studying the directed graphs in which they are embedded. A first step would be to study all vertices that can be reached from the non-trivial strongly components. This would illuminate connections between the SCCs and expose the fractal structure of the directed graph, which is not observed when only studying the SCCs themselves.
    \item Another natural next step is to study the model under weaker moment conditions. The first condition to eliminate would be $\E\left[(D^-)^i(D^+)^j\right]<\infty$ for $(i,j)=(1,3)$ and $(i,j)=(3,1)$. Removing the former condition would in some sense make the identifications less uniform on the ancestral lines. To be precise, $(\hat{H}^\ell(k)/\hat{H}(k),k\geq 0)$ will not necessarily converge to a constant process under rescaling of time, which means that the in-edges that can be used to form surplus edges are spread out less uniformly on the out-components. We have reason to believe that this would place the model in  a different universality class, but further research is needed to confirm this. Removing the latter condition requires an adaptation of the proof of Proposition \ref{prop.anomalousedges} that does not use the Cauchy-Schwarz inequality. Also the heavy-tailed case is not well-understood, but given our results, it is natural to expect that a potential limit object would be embedded in a tilted stable tree as defined in \cite{conchon--kerjanStableGraphMetric2020}. Moreover, one could define hybrid models by letting the tail-behaviour of the in- and out-degrees be different. 
    \item We conjecture that the rank-1 inhomogeneous directed random graph model under suitable conditions is part of the same universality class as the directed Erd\H{o}s-Rényi graph \cite{goldschmidtScalingLimitCritical2019} and the model we consider in this work. We believe that our methods and the methods of \cite{goldschmidtScalingLimitCritical2019} can be adapted to obtain a metric space scaling limit for the inhomogeneous directed random graph model, and we intend to pursue this in future work. 

\end{enumerate}