\begin{abstract}
  We consider the strongly connected components (SCCs) of a uniform directed graph on $n$ vertices with i.i.d. in- and out-degree pairs distributed as $(D^-,D^+)$, with $\E[D^+]=\E[D^-]=\mu$. We condition on equal total in-degree and total out-degree. A phase transition for the emergence of a giant strongly connected component, that contains a positive proportion of the vertices, is known to occur at the critical value $\E[D^-D^+]/\mu=1$. We study the model at this critical value and, additionally, require that $\E[(D^-)^i(D^+)^j]<\infty$ for all $i+j\leq 3$, and for $(i,j)=(1,3)$, or $(i,j)=(3,1)$. We show that, under these conditions, the strongly connected components ranked by decreasing size with distances rescaled by $n^{-1/3}$ converge in distribution to a sequence of finite strongly connected directed multigraphs with edge lengths that are either $3$-regular or loops. The limit objects lie in a $3$-parameter family, which contains the scaling limit of the SCCs in the directed Erd\H{o}s-R\'enyi model at criticality as found by Goldschmidt and Stephenson (2019). This is the first universality result for the scaling limit of a critical directed graph model. Moreover, it is the first quantitative result on the directed configuration model at criticality. As a trivial consequence, the largest strongly connected components at criticality contain $\Theta(n^{1/3})$ vertices and edges in probability, and the diameter of the directed graph at criticality is $\Omega(n^{1/3})$ in probability, in contrast to the non-critical cases; two results that were previously unknown.  We use a metric on the space of such multigraphs in which two multigraphs are close if there are compatible isomorphisms between their vertex and edge sets which roughly preserve the edge lengths. We use the product topology on the sequence of multigraphs. Our method of proof involves a depth-first exploration of the directed graph, resulting in a spanning forest with additional identifications, of which we study the limit under rescaling.
\end{abstract}